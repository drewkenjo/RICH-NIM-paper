\section{Maroc chip calibration}

To remove the non-linearity within the ADC readout, a procedure was developed to convert the amplitude of the MAROC slow shaper signal from ADC channels into charge. The MAROC has a built-in charge injection functionality consisting of a test input pin which is connected to the preamplifiers through a logic network of switches and 2 pF capacitors. Together with an external step function generator, this can be used to inject a controllable amount of charge directly into the preamplifiers. We measured the output of the slow shaper in ADC channels for 82 different input charges ranging from 0 to 4 pC. Fig.~\ref{fig:MAROCcalib} shows the relationship between the injected charge and the measured amplitude in units of ADC channels for three different readout channels. The relationship between charge and ADC channels is linear up to about 1.5 pC. This distribution was observed to vary between chips and pixels, and thus a unique distribution was measured for all 64 pixels on each MAROC used in this study. 

\begin{figure}[hbt]
	\centering
	\includegraphics[width=\linewidth]{figures/adc_v_charge.png}
	\caption{Response of MAROC slow shaper in ADC channels as a function of the injected charge. This curve is for MAROC id 726 and pixel 01.}
	\label{fig:MAROCcalib}
\end{figure}

This calibration data was used to convert the measured amplitude in ADC channels into charge collected on an event-by-event basis. The interpolation is done by assigning quadratic functions to every calibration data point to obtain local estimates of the charge as a function of ADC channels. The quadratic functions are obtained by fitting the nearest 5 data points, so that the value of the function is not constrained to equal the measured ADC channels at the given injected charge. For each event, the charge is estimated as a linearly weighted average between the extrapolations of the quadratic functions from the nearest two data points. Fig.~\ref{fig:H12700calib} and Fig.~\ref{fig:H8500calib} show typical amplitude distributions before and after this conversion is applied for one H12700 MAPMT pixel and one H8500 MAPMT pixel, respectively. For both, the conversion to charge extends the high-amplitude tails of the spectra due to the nonlinearity of the ADC readout.

\begin{figure}[hbt]
	\centering
	\includegraphics[width=\linewidth]{figures/GA0982_w1_g064_v1100_063_adc.png}
	\includegraphics[width=\linewidth]{figures/GA0982_w1_g064_v1100_063_pC.png}	
	\caption{Top: A typical spe spectrum for one HA12700 pixel in units of ADC channels. Bottom: The same spectrum after converting the units into pC.}
	\label{fig:H12700calib}
\end{figure}

\begin{figure}[hbt]
	\centering
	\includegraphics[width=\linewidth]{figures/CA7811_w1_g064_v1100_056_adc.png}
	\includegraphics[width=\linewidth]{figures/CA7811_w1_g064_v1100_056_pC.png}	
	\caption{Top: A typical spe spectrum for one HA8500 pixel in units of ADC channels. Bottom: The same spectrum after converting the units into pC.}
	\label{fig:H8500calib}
\end{figure}