\section{Conclusion}

As a part of CLAS12 RICH detector upgrade at Jefferson Lab, we have conducted a mass study of 399 H12700 MAPMTs from Hamamatsu, with the goal to evaluate every tube and characterize every pixel in terms of their gain, quantum efficiency, crosstalk contribution, and optimized threshold for detecting single Cherenkov photons. The dedicated test setup included a precision picosecond laser, gears for the relative positioning of the MAPMTs in the setup, RICH detector front-end electronics, and fully automated data acquisition and control systems. The non-linearity of the data acquisition, the ADC-to-charge conversion calibration parameters of every channel, and the absolute calibration of the number of laser photons reaching every pixel in every event were measured in special separate experiments. The bulk measurements consisted of six expositions of every group of six MAPMTs at three levels of low light and two applied high voltages, 1000~V, and 1100~V. The systematic uncertainties dependent on the MAPMT placement in the group of six were evaluated and found to be within the final parameter uncertainties.

In a set of dedicated detailed studies we observed and quantified the pixel-to-pixel signal crosstalk using a two dimensional amplitude distribution analysis. Using several representative MAPMTs of both types we found that the H8500 model is characterized by quite significant amplitude spectral contributions to a given pixel from its neighbors in the matrix, with such crosstalk contributions reaching up to 50\% of the spectral amplitude. At the same time, the crosstalk in H12700 MAPMTs was generally less than about 3-5\%. Methods of separating and taking into account the crosstalk contributions to the amplitude distributions from any pixel were developed, using the two dimensional analysis, and also approximating and evaluating the contributions based on the spectral shape using the computational model. The first approach is applicable to all MAPMTs studied, but it is labor intensive and works correctly only in the conditions of extremely low light in the tests. The second approach works well for the H12700 MAPMTs, and was used for the bulk measurements.

The accumulated amplitude spectra were corrected to the non-linearity of the data acquisition, and converted to the calibrated total charge distributions. The recently published state-of-the-art computational model, describing photon detector response functions measured in conditions of low light, was extended to include the successful description of the crosstalk contributions to the spectra from the neighboring pixels. The updated model was used to parameterize and extract the SPE response functions of every pixel, and characterize its properties such as gain, quantum efficiency, and crosstalk, and to determine the optimal signal threshold values to evaluate its efficiency to Cherenkov photons. The stability and reproducibility of the extracted parameter values were verified by the comparison of the six independent measurements of each pixel, allowing us to evaluate the uncertainties in the measurements of the major model parameters. One of the extracted parameters, the average multiplication of a photoelectron on the first dynode $\nu$ was found significantly larger on the H12700 compared to the H8500 MAPMTs. That difference corresponds to the resulting difference between the SPE efficiency of the two models.  That observation, together with much smaller crosstalk contributions, generally confirms our early decision to switch to the H12700 as the MAPMT of choice for the RICH detector.

The database of extracted parameters will be used for the final selection and arrangement of the MAPMTs in the new RICH detector, and for determining their optimal operation parameters, such as operating high voltage, gain, and threshold of the front-end electronics. The results show that the quality of the H12700 MAPMT mass production at Hamamatsu is high, satisfying our needs in the good position-sensitive single photoelectron detectors.



\section{Acknowledgements}
This material is based upon work supported by the U.S. Department of Energy, Office of Science, Office of Nuclear Physics under contract DE-AC05-06OR23177, and in part by DOE Grant No. DE-FG02-04ER41309.
We sincerely thank 
Bishnu Karki,
Charles Hanretty,
Aiden Boyer,
Chris  Cuevas,
Matteo Turisini, and
Carl Zorn
for their technical support and help in taking and analyzing the experimental data. We would
also like to thank 
Patrizia Rossi, 
Marco Contalbrigo, 
Marco Mirazita,
Fatiha Benmokhtar,
Kyungseon Joo,
and Zhiwen Zhao
for the fruitful discussions, continuous support and interest in this work. Special thanks to Danial Carman for proofreading the article and making valuable comments.