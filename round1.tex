\documentclass[11pt]{report}

\usepackage{amsmath,amsfonts,amssymb}
\usepackage{graphicx}
\usepackage[colorlinks=true, allcolors=blue]{hyperref}
\usepackage{listings}
\usepackage{tcolorbox}
\usepackage[top=1in,bottom=1in,left=0.5in,right=0.5in]{geometry}
\usepackage{graphicx}
\usepackage{authblk}
\usepackage{amsmath, amsfonts, graphicx, placeins}
\usepackage{amssymb}% for \sphericalangle
\usepackage{braket}
\usepackage{subcaption}
\usepackage{mathtools}
\usepackage{breqn}
\usepackage{hyperref}
\usepackage{rotating}
\usepackage{multirow}
\usepackage{color}
\usepackage{placeins}
\usepackage{todonotes}


\lstset{ 
basicstyle=\linespread{0.8}\scriptsize,
commentstyle=\color{gray},
frame=single,
numbers=left,
tabsize=2
}


\begin{document} 

{\LARGE\centering
Review notes for “Characterization of Multianode Photomultiplier Tubes for use in the CLAS12 RICH Detector”\\[1cm]

June 2022\\[1cm]
}

\begin{tcolorbox}[enlarge top by=2em,colbacktitle=black!60!white,colframe=black!80!white,left=0pt,right=0pt,top=0pt,bottom=0pt,boxrule=0.3pt,title=\bfseries1.01]
Line 39: “With a high quantum efficiency in the visible light region” – it might be good to pro-vide a number (for example, 30\% at 450 nm, or something like that) here because ‘high’ can be relative.
\end{tcolorbox}

{\bfseries we addressed the comment in the text:}

\includegraphics[width=\linewidth]{round1/1.01.png}

\begin{tcolorbox}[enlarge top by=2em,colbacktitle=red!60!white,colframe=black!80!white,left=0pt,right=0pt,top=0pt,bottom=0pt,boxrule=0.3pt,title=\bfseries1.02]
Fig  10+11:  It  is  difficult  to  clearly  observe  the  various  features  in  the  plots.   Additionally,  there  is no indication in the Figure caption which side of the dashed red line is being cut.  The central figure should have units on the y-axis (counts or arb.  units).  The central figure should probably have the same x-axis range as the other figures in the panel.  Similarly for Fig 11.
\end{tcolorbox}

\begin{tcolorbox}[enlarge top by=2em,colbacktitle=black!60!white,colframe=black!80!white,left=0pt,right=0pt,top=0pt,bottom=0pt,boxrule=0.3pt,title=\bfseries1.03]
Line 234:  Where does this tabulated efficiency of the photodiode come from?  Are there uncertain-ties on this number?
\end{tcolorbox}

{\bfseries we addressed the comment in the text:}

\includegraphics[width=\linewidth]{round1/1.03.png}

\begin{tcolorbox}[enlarge top by=2em,colbacktitle=black!60!white,colframe=black!80!white,left=0pt,right=0pt,top=0pt,bottom=0pt,boxrule=0.3pt,title=\bfseries1.04]
Fig  12  caption:  Needs  a  bit  of  rewriting,  its  currently  poorly  worded.   For  example:The light intensity distribution dN/dS, defined as ..., for a row of three MaPMTs in the laser stand.
\end{tcolorbox}

{\bfseries we addressed the comment in the text:}

\includegraphics[width=\linewidth]{round1/1.04.png}

\begin{tcolorbox}[enlarge top by=2em,colbacktitle=black!60!white,colframe=black!80!white,left=0pt,right=0pt,top=0pt,bottom=0pt,boxrule=0.3pt,title=\bfseries1.05]
Line 253:  Issue with math mode?  Why are the units italicized?
\end{tcolorbox}

{\bfseries we addressed the comment in the text.}

\begin{tcolorbox}[enlarge top by=2em,colbacktitle=black!60!white,colframe=black!80!white,left=0pt,right=0pt,top=0pt,bottom=0pt,boxrule=0.3pt,title=\bfseries1.06]
Line 263:  Are there other efficiencies to consider?  For example, what is the collection efficiency (the  efficiency  for  successfully  converting  the  generated  PE  at  the  photocathode  to  a  signal  at  the anode – which might be non-unity due to inefficiencies collecting the first PE at the first dynode).
\end{tcolorbox}

{\bfseries we addressed the comment in the text:}

\includegraphics[width=\linewidth]{round1/1.06.png}

\begin{tcolorbox}[enlarge top by=2em,colbacktitle=black!60!white,colframe=black!80!white,left=0pt,right=0pt,top=0pt,bottom=0pt,boxrule=0.3pt,title=\bfseries1.07]
Line 287:  I do not think the parameters, such as scale and gain, need to be italicized.
\end{tcolorbox}

{\bfseries we addressed the comment in the text.}

\begin{tcolorbox}[enlarge top by=2em,colbacktitle=black!60!white,colframe=black!80!white,left=0pt,right=0pt,top=0pt,bottom=0pt,boxrule=0.3pt,title=\bfseries1.08]
Line  353:   The  fits  look  very  impressive.   It  could  be  worth  highlighting  that  by  utilizing  this method, the model and associated parameters can  be used to simulate the response of these MaPMTs in a MC simulation of the detector.
\end{tcolorbox}

{\bfseries we addressed the comment in the text:}

\includegraphics[width=\linewidth]{round1/1.08.png}

\begin{tcolorbox}[enlarge top by=2em,colbacktitle=black!60!white,colframe=black!80!white,left=0pt,right=0pt,top=0pt,bottom=0pt,boxrule=0.3pt,title=\bfseries1.09]
Line 447:  There are some features of the QE vs.  pixel number plot, is this understood?
\end{tcolorbox}

{\bfseries we addressed the comment in the text:}

\includegraphics[width=\linewidth]{round1/1.09.png}

\begin{tcolorbox}[enlarge top by=2em,colbacktitle=red!60!white,colframe=black!80!white,left=0pt,right=0pt,top=0pt,bottom=0pt,boxrule=0.3pt,title=\bfseries1.10]
Figure 21:  I think you can remove the word ‘Colors’ from the plot.  Same for all plot with ‘Colors’in the legend.
\end{tcolorbox}

{\bfseries todo}
reply that we don't want to remove Colors


\begin{tcolorbox}[enlarge top by=2em,colbacktitle=black!60!white,colframe=black!80!white,left=0pt,right=0pt,top=0pt,bottom=0pt,boxrule=0.3pt,title=\bfseries1.11]
Figure 24:  You can remove ‘essentially on top of each other’ from the caption.  I’m not certain I understand why “For the data collected at wheel position 3,  this ratio is the quantum efficiency of the individual pixels.” Is this because you’ve performed the conversion only that that wheel position? Why do you chose to do that only for wheel position 3?  Does the Q.E agree for all wheel positions?
\end{tcolorbox}

{\bfseries we addressed the comment in the text:}

\includegraphics[width=\linewidth]{round1/1.11.png}

\begin{tcolorbox}[enlarge top by=2em,colbacktitle=black!60!white,colframe=black!80!white,left=0pt,right=0pt,top=0pt,bottom=0pt,boxrule=0.3pt,title=\bfseries1.12]
Line 550:  State clearly the value that you measure and the value that the manufacturer measurement.  Can you comment on why they might differ?
\end{tcolorbox}

{\bfseries we addressed the comment in the text:}

\includegraphics[width=\linewidth]{round1/1.12.png}

\begin{tcolorbox}[enlarge top by=2em,colbacktitle=black!60!white,colframe=black!80!white,left=0pt,right=0pt,top=0pt,bottom=0pt,boxrule=0.3pt,title=\bfseries1.13 and 1.14]
Line 557:  the word inefficiency does not need to be in quotes.

Line 599 and elsewhere:  I do not think quantum efficiency should be capitalized.
\end{tcolorbox}

{\bfseries we addressed the comment in the text.}







\clearpage






%%%%%%%%%%%%%%%%%%%%%%%%%%%%%%%%%%%%%
%%%%%%%%%%%%%%%%%%%%%%%%%%%%%%%%%%%%%
\begin{tcolorbox}[enlarge top by=2em,colbacktitle=red!60!white,colframe=black!80!white,left=0pt,right=0pt,top=0pt,bottom=0pt,boxrule=0.3pt,title=\bfseries2.01]
The paper is written as a long experimental note and is characterized by a pronounced verbosity. As a general comment the paper is quite difficult to read.
\end{tcolorbox}

\begin{tcolorbox}[enlarge top by=2em,colbacktitle=green!60!white,colframe=black!80!white,width=0.9\linewidth,left=30pt,right=30pt,top=10pt,bottom=10pt,boxrule=0.3pt,title=\bfseries our draft remarks]
{\bfseries Pavel}
Apart from being an experimental note, this work introduces the new extension of the calculational model describing the PMT response to few-photoelectron events, which takes into account the description and characterization of the effect of the signal cross-talk in Multianode PMTs. The new method of selection between the true events and the events when a signal is caused by the hit in a neighboring pixel will help improve the spatial resolution of single photoelectron detectors. 
\end{tcolorbox}


%%%%%%%%%%%%%%%%%%%%%%%%%%%%%%%%%%%%%
%%%%%%%%%%%%%%%%%%%%%%%%%%%%%%%%%%%%%
\begin{tcolorbox}[enlarge top by=2em,colbacktitle=red!60!white,colframe=black!80!white,left=0pt,right=0pt,top=0pt,bottom=0pt,boxrule=0.3pt,title=\bfseries2.02]
A list of specifications and requirements on the PMTs coming from the CLAS12 RICH are not clearly defined at the beginning of the paper, therefore the reader must guess which are the most important parameters to be measured and what characteristics will define a "good" PMT for the CLAS12 RICH production
\end{tcolorbox}

\begin{tcolorbox}[enlarge top by=2em,colbacktitle=green!60!white,colframe=black!80!white,width=0.9\linewidth,left=30pt,right=30pt,top=10pt,bottom=10pt,boxrule=0.3pt,title=\bfseries our draft remarks]
{\bfseries Valery}
I agree with the statement. Such list needs to be produced.

{\bfseries Valery}
We made detailed characterization of around 400 H12700 MaPMTs, as well as several H8500 to make a comparison of the two models. These data give us the possibilities to better understand the performance of the first RICH where we are using both MaPMT models.

The single photoelectron spectra were measured for each pixel at different high voltages and light intensities of the laser test setup. Using the standard for RICH dedicated front-end electronics the setup allowed us to characterize each pixel’s properties such as gain, quantum efficiency, signal crosstalk between neighboring pixels, and determine the signal threshold values to optimize their efficiency to detect Cherenkov photons. These parameters were determined for each pixel out of 400 MaPMTs what gives us the possibility to select the best MaPMTs and determine the working parameters of the front-end electronic in the real experiment. The results of this study are presented in this paper.

{\bfseries Valery}
note that few MAPMTs were sent back due to low gain


\includegraphics[width=\linewidth]{round1/2.02.png}
\end{tcolorbox}


%%%%%%%%%%%%%%%%%%%%%%%%%%%%%%%%%%%%%
%%%%%%%%%%%%%%%%%%%%%%%%%%%%%%%%%%%%%
\begin{tcolorbox}[enlarge top by=2em,colbacktitle=red!60!white,colframe=black!80!white,left=0pt,right=0pt,top=0pt,bottom=0pt,boxrule=0.3pt,title=\bfseries2.03]
The type of MaPMTs studied in this paper (Hamamatsu H12700) have been characterized and described in the literature and are characterized by a large pixel-by-pixel and tube-by-tube gain variation (with a typical spread of 1 to 3). In addition, this type of MaPMT has some typical features of the individual pixel response which depend on the pixel position inside the matrix. Indication of the "pixel number" appears in the text and in specific plots (starting from section 4), and a channel map with channel numbers should be introduced to let the reader fully understand the measurements
\end{tcolorbox}

\begin{tcolorbox}[enlarge top by=2em,colbacktitle=green!60!white,colframe=black!80!white,width=0.9\linewidth,left=30pt,right=30pt,top=10pt,bottom=10pt,boxrule=0.3pt,title=\bfseries our draft remarks]
I agree with the statement. The PMT channel map information might be there already, just need to highlight or give it more stress/explanation in the text.
\end{tcolorbox}


%%%%%%%%%%%%%%%%%%%%%%%%%%%%%%%%%%%%%
%%%%%%%%%%%%%%%%%%%%%%%%%%%%%%%%%%%%%
\begin{tcolorbox}[enlarge top by=2em,colbacktitle=red!60!white,colframe=black!80!white,left=0pt,right=0pt,top=0pt,bottom=0pt,boxrule=0.3pt,title=\bfseries2.04]
This MaPMT type is also affected by after-pulses, which might have a strong impact for the CLAS12 RICH operations and not mentioned at all in the paper
\end{tcolorbox}

\begin{tcolorbox}[enlarge top by=2em,colbacktitle=green!60!white,colframe=black!80!white,width=0.9\linewidth,left=30pt,right=30pt,top=10pt,bottom=10pt,boxrule=0.3pt,title=\bfseries our draft remarks]

{\bfseries Pavel}
Need to find a reference mentioning after-pulses and explain that such study was not considered important because the RICH signals are not in trigger and the delayed pulses do not interfere with our data. They may increase the accidental noise which is measured separately in the detector. By the way, I guess it can be observed in RICH-1 by plotting rates vs. time agter the beam stop? 

{\bfseries note to editor:}

- no immediate afterpulses (include scope picture)

- delayed afterpulses are not important to us

{\bfseries Pavel}
Maybe include in the beginning somewhere: "During our tests we have not  seen any effects of the immediate after-pulses observed by other authors (see, for example, Ref.~[]). No oscilloscope traces, no instabilities in the amplitude spectra showed their presence. The long-delayed after-pulses are not critical for the RICH in which only time-synchronized events are accepted."

\includegraphics[width=0.7\linewidth]{round1/afterpulses.png}

\end{tcolorbox}



%%%%%%%%%%%%%%%%%%%%%%%%%%%%%%%%%%%%%
%%%%%%%%%%%%%%%%%%%%%%%%%%%%%%%%%%%%%
\begin{tcolorbox}[enlarge top by=2em,colbacktitle=red!60!white,colframe=black!80!white,left=0pt,right=0pt,top=0pt,bottom=0pt,boxrule=0.3pt,title=\bfseries2.05]
All the main results of the paper refer to a computational model, described in detail in Ref. [12], which is adapted to the experimental conditions described in the paper (in section 6 and in the appendix). This is the part of the paper which is the most difficult to read and understand. Multiple citations of Ref. [12] are made in the text, and the reader is forced to go to that reference to fully understand in detail the procedure described in the paper.
\end{tcolorbox}

\begin{tcolorbox}[enlarge top by=2em,colbacktitle=green!60!white,colframe=black!80!white,width=0.9\linewidth,left=30pt,right=30pt,top=10pt,bottom=10pt,boxrule=0.3pt,title=\bfseries our draft remarks]
{\bfseries Pavel}
That is a correct statement. The computational model describes the amplitude distributions of signals coming from every PMT pixel in the conditions of low light and allows to extract its important physical parameters. Such as its amplification gain in pC per one photoelectron, the average multiplicity of the observed photoelectrons in given conditions (which can be converted to quantum efficiency), approximate shape of the single-photoelectron spectrum. All this can be done independently of the test conditions. The new extension of the model allows to evaluate the cross-talk effects between neighboring pixels and evaluate the optimal amplitude to separate between them. As it is now, the model is quite complicated, and to fully understand the outcome the familiarity with Ref. [12] is required. 

{\bfseries note to editor:}

- it is a complicated subject
- impossible to reply because there is no question
\end{tcolorbox}



%%%%%%%%%%%%%%%%%%%%%%%%%%%%%%%%%%%%%
%%%%%%%%%%%%%%%%%%%%%%%%%%%%%%%%%%%%%
\begin{tcolorbox}[enlarge top by=2em,colbacktitle=red!60!white,colframe=black!80!white,left=0pt,right=0pt,top=0pt,bottom=0pt,boxrule=0.3pt,title=\bfseries2.06]
Concerning plots, many of them have captions/labels too small to be easily visible by the reader
\end{tcolorbox}

\begin{tcolorbox}[enlarge top by=2em,colbacktitle=green!60!white,colframe=black!80!white,width=0.9\linewidth,left=30pt,right=30pt,top=10pt,bottom=10pt,boxrule=0.3pt,title=\bfseries our draft remarks]
	
{\bfseries Pavel}
We would refer this observation to the Publisher. We tried to make the size of the labels no smaller than they were published in Ref. [12], but please let us know if it’s not acceptable 
\end{tcolorbox}



%%%%%%%%%%%%%%%%%%%%%%%%%%%%%%%%%%%%%
%%%%%%%%%%%%%%%%%%%%%%%%%%%%%%%%%%%%%
\begin{tcolorbox}[enlarge top by=2em,colbacktitle=red!60!white,colframe=black!80!white,left=0pt,right=0pt,top=0pt,bottom=0pt,boxrule=0.3pt,title=\bfseries2.07]
The fit parameter results show some inconsistencies in Fig. 14-15-16 (especially the mu parameter in the top-right plot in each figure).
\end{tcolorbox}

\begin{tcolorbox}[enlarge top by=2em,colbacktitle=green!60!white,colframe=black!80!white,width=0.9\linewidth,left=30pt,right=30pt,top=10pt,bottom=10pt,boxrule=0.3pt,title=\bfseries our draft remarks]
{\bfseries Pavel}
Fig. 14 is not consistent with 15 and 16 as it is done with different MaPMT.
There is actually a good consistency between Figs. 15 and 16. The HVs are different, and the major change is in the "scale" parameter, with mu staying the same. The statistical spead of other parameters is expected.
The differences between the plots (a)-(d) in the Figs. 14-15-16 are discussed in the captions and in the text. Mu parameters depend on the type of data (3mm or 6mm mask, open pixel with software crosstalk removal, and model approximation). In case of H8500 (Fig. 14) the model cannot extract the crosstalk correctly because it's too wide.
\end{tcolorbox}


%%%%%%%%%%%%%%%%%%%%%%%%%%%%%%%%%%%%%
%%%%%%%%%%%%%%%%%%%%%%%%%%%%%%%%%%%%%
\begin{tcolorbox}[enlarge top by=2em,colbacktitle=black!60!white,colframe=black!80!white,left=0pt,right=0pt,top=0pt,bottom=0pt,boxrule=0.3pt,title=\bfseries2.08]
The chi2 of the 4 fits presented in Fig. 18 is very high, which indicates that probably the model does not represent the data well for the "highest light intensity" configuration. The illumination conditions are described in different ways that confuse the reader: a precise and unique definition should be used throughout the text
\end{tcolorbox}

\includegraphics[width=\linewidth]{round1/2.08.png}



%%%%%%%%%%%%%%%%%%%%%%%%%%%%%%%%%%%%%
%%%%%%%%%%%%%%%%%%%%%%%%%%%%%%%%%%%%%
\begin{tcolorbox}[enlarge top by=2em,colbacktitle=red!60!white,colframe=black!80!white,left=0pt,right=0pt,top=0pt,bottom=0pt,boxrule=0.3pt,title=\bfseries2.09]
The computational model returns a set of parameters after the fit which should be related to the "physical observables" typical of a PMT, like e.g. quantum efficiency (QE) or gain. Most of the "measurements" presented in the paper are the result of a fit using the above-mentioned model, and are not compared with independent measurements, which would be necessary to assess the correctness of the model. Only self-consistency checks of the model are partially shown in the paper. Just to quote a few examples:
\end{tcolorbox}

\begin{tcolorbox}[enlarge top by=2em,colbacktitle=green!60!white,colframe=black!80!white,width=0.9\linewidth,left=30pt,right=30pt,top=10pt,bottom=10pt,boxrule=0.3pt,title=\bfseries our draft remarks]
	
{\bfseries Pavel}
That is a tough one. Develop another model and measure the same parameters to compare… Not clear what would constitute an “independent measurement”. Should we stress again that every pixel was evaluated in six independent measurements with consistent results. Plus the systematic error studies.

{\bfseries note to editor:}

- statistical errors are there (very small)

- explain in details that the reviewer misunderstands the plots

- no independent measurements can reasonably be performed

- mention "quasi model independent" extraction of mu/gain for 1spe spectra and their comparison with Pavel's model

{\bfseries Pavel}

\includegraphics[width=\linewidth]{round1/2.09.png}
\end{tcolorbox}


%%%%%%%%%%%%%%%%%%%%%%%%%%%%%%%%%%%%%
%%%%%%%%%%%%%%%%%%%%%%%%%%%%%%%%%%%%%
\begin{tcolorbox}[enlarge top by=2em,colbacktitle=black!60!white,colframe=black!80!white,left=0pt,right=0pt,top=0pt,bottom=0pt,boxrule=0.3pt,title=\bfseries2.09a]
The top-right plot in Fig. 19 shows QE measurements as a function of the anode number for a PMT. No error bars (except probably the ones on the fit parameters) are indicated to understand the precision in the estimate of this important quantity (no evaluation of systematic errors). A comparison with an independent measurement would be important, even on a limited number of MaPMTs (e.g. measurement of the absolute QE made using a calibrated monocromatic light source)
\end{tcolorbox}

\includegraphics[width=\linewidth]{round1/2.09a.png}



%%%%%%%%%%%%%%%%%%%%%%%%%%%%%%%%%%%%%
%%%%%%%%%%%%%%%%%%%%%%%%%%%%%%%%%%%%%
\begin{tcolorbox}[enlarge top by=2em,colbacktitle=black!60!white,colframe=black!80!white,left=0pt,right=0pt,top=0pt,bottom=0pt,boxrule=0.3pt,title=\bfseries2.09b]
The top-left plot in Fig. 19 shows the scale (proportional to the PMT anode gain) as a function of the anode number for a PMT. Some variation is shown, still without error bars (or with error bars only coming from the fit parameters, with no systematic error evaluation). However, a larger pixel-to-pixel variation is expected for this type of PMT
\end{tcolorbox}

\includegraphics[width=\linewidth]{round1/2.09b.png}

\includegraphics[width=\linewidth]{round1/2.09c.png}



%%%%%%%%%%%%%%%%%%%%%%%%%%%%%%%%%%%%%
%%%%%%%%%%%%%%%%%%%%%%%%%%%%%%%%%%%%%
\begin{tcolorbox}[enlarge top by=2em,colbacktitle=red!60!white,colframe=black!80!white,left=0pt,right=0pt,top=0pt,bottom=0pt,boxrule=0.3pt,title=\bfseries2.10]
In conclusion, the paper presents measurements on a large PMT population (about 400 detectors), which are essentially based on a computational model described in another paper (Ref. [12]).
\end{tcolorbox}

\begin{tcolorbox}[enlarge top by=2em,colbacktitle=green!60!white,colframe=black!80!white,width=0.9\linewidth,left=30pt,right=30pt,top=10pt,bottom=10pt,boxrule=0.3pt,title=\bfseries our draft remarks]
{\bfseries Pavel}
Again, it’s not only that. Novel extension of the model to evaluate and parameterize cross-talk. Absolute gain calibration and QE calibration for every pixel are state of the art.  
\end{tcolorbox}



%%%%%%%%%%%%%%%%%%%%%%%%%%%%%%%%%%%%%
%%%%%%%%%%%%%%%%%%%%%%%%%%%%%%%%%%%%%
\begin{tcolorbox}[enlarge top by=2em,colbacktitle=red!60!white,colframe=black!80!white,left=0pt,right=0pt,top=0pt,bottom=0pt,boxrule=0.3pt,title=\bfseries2.11]
In its current form the paper does not satisfy the Journal requirements and should not be published. A major revision is needed, strongly reducing the written text to improve conciseness and reducing the number of plots, inserting only representative ones. What is missing in particular is the list of specifications on the PMTs coming from the physics of the experiment, and the requirements that the PMTs must satisfy to pass the "selection" to be mounted inside the second RICH of CLAS12, as the reader might expect reading the title.
\end{tcolorbox}

\begin{tcolorbox}[enlarge top by=2em,colbacktitle=green!60!white,colframe=black!80!white,width=0.9\linewidth,left=30pt,right=30pt,top=10pt,bottom=10pt,boxrule=0.3pt,title=\bfseries our draft remarks]
{\bfseries Pavel}
We respectfully disagree with the request to strongly reduce the written text and number of plots. We believe such change would make the work less comprehensible.  But we agree that the list of the PMT specifications and requirements should be collected in one place in the beginning of the paper.
\end{tcolorbox}

%%%%%%%%%%%%%%%%%%%%%%%%%%%%%%%%%%%%%
%%%%%%%%%%%%%%%%%%%%%%%%%%%%%%%%%%%%%
\begin{tcolorbox}[enlarge top by=2em,colbacktitle=red!60!white,colframe=black!80!white,left=0pt,right=0pt,top=0pt,bottom=0pt,boxrule=0.3pt,title=\bfseries2.12]
In addition, the validation of the model using independent data on individual parameters is needed to assess the precision of the used method (just to quote an example, a comparison of QE with independently measured values, e.g. by Hamamatsu or by a calibrated monocromatic light source), which lacks evaluation of systematic uncertainties on the presented parameters.
\end{tcolorbox}

\begin{tcolorbox}[enlarge top by=2em,colbacktitle=green!60!white,colframe=black!80!white,width=0.9\linewidth,left=30pt,right=30pt,top=10pt,bottom=10pt,boxrule=0.3pt,title=\bfseries our draft remarks]
{\bfseries Pavel}
I believe the comparison of the QE with Hamamatsu data is given in the paper. We will try to explain it more clearly in the paper. Not sure about the precision comparison of the anode gains with Hamamatsu, but it’s done using different techniques and convoluted with QE, and done with the different light, so it’s difficult numerically. But we can say that qualitatively the agreement is good.
\end{tcolorbox}



\clearpage




%%%%%%%%%%%%%%%%%%%%%%%%%%%%%%%%%%%%%
%%%%%%%%%%%%%%%%%%%%%%%%%%%%%%%%%%%%%
\begin{tcolorbox}[enlarge top by=2em,colbacktitle=red!60!white,colframe=black!80!white,left=0pt,right=0pt,top=0pt,bottom=0pt,boxrule=0.3pt,title=\bfseries Addition 1]
\end{tcolorbox}

\includegraphics[width=\linewidth]{round1/add1.png}

\end{document} 
