\section{Conclusion}
We have tested several hundreds of the Hamamatsu H12700 and H8500 multi-anode photomultiplier tubes in order to evaluate the application of these MAPMTs in the Ring Imaging Cherenkov detectors. The PMT’s performance study was done using the laser stand with low light illumination to simulate one-photoelectron regime as expected in the real experiment. We were using the state-of-the-art mathematical model for the characterization of the photomultipliers that were design to describe the PMT’s signals with low number of photoelectrons. The model was modified to take into account the effects of the cross-talk between the neighboring pixels. The stability and consistency of the parameter values measured in different laser beam intensities and at different applied high voltages allows to characterize the intrinsic features of every pixel, independent on the measurement conditions. 
We compared the following parameters of two types of MAPMT: H12700 and H8500
\begin{itemize}
\item The shapes of the one photoelectron spectra
\item PMT's gain as a function of high voltage
\item Amplification of the first dynode
\item Efficiency to detect one-photoelectron signal
\item Detailed comparison of the cross-talk characteristics
\item Quantum efficiency of each pixel
\end{itemize}
The large amount of PMT’s gives us the possibility to do statistical analysis of all parameters in hands and compare the performance of H12700 and H8500 for an application in high energy and nuclear physics. The parameter database accumulated as the result of this work may be used for the selection of the MAPMTs for installation in the RICH detector, and for the optimization of the future run parameters, such as the tube placement selection, setting the values of operating high voltage, electronics gains and thresholds in the detector. The data also provide the opportunity to evaluate the spread of such parameters in the mass production of the MAPMT devices as the channel gains, quantum efficiencies, SPE spectral shapes, parameters of the cross-talk, across the face of each tube, and across the whole set. The results show that the quality of MAPMT mass production at Hamamatsu is high and satisfies our needs in the good quality single photoelectron detection.
 
