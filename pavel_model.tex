\section{Mathematical model for the description of the PMT's response}

The goal for using MAPMTs in RICH detectors is to achieve reliable detection of single photons in the Cherenkov light radiation cones. Single photon incident on a PMT face may knock out a single photoelectron from the PMT's photocathode with a certain probability, defined as Quantum Efficiency (QE). The photoelectrons cascade inside the PMT to generate a typical amplified electrical signal at the anode. The amplitude distribution of the single photoelectron (SPE) signals depends on the PMT design and high voltage applied, and varies from pixel to pixel. Tests and characterization of multiple MAPMTs include measuring the SPE amplitude distributions for every pixel, finding out the appropriate amplitude thresholds, and determining QE. To achieve this goal we used the methods developed in ~\cite{DEGTIARENKO20171}, expanded to include the new empirical method to take into account the effects of pixel-to-pixel cross talk in the H12700 tubes. The reference ~\cite{DEGTIARENKO20171} describes in detail the mathematical model used to extract and parametrize the SPE distributions from the measurements using the test setup. The method allows in principle to describe the SPE functions of essentially any complexity by decomposing them as a sum of Poisson distributions with different averages. For the detailed explanations and the definition of the model parameters please refer to ~\cite{DEGTIARENKO20171}. The list of main parameters includes $\mu$, average number of photoelectrons produced by the laser in a given pixel per one pulse, and {\it{scale}}, average amplitude of the SPE distribution in pC. Five parameters determine the shape of the SPE distribution, defined as a normalized sum of three Poisson distributions with different average multiplication coefficients applied to the photoelectron on the first dynode of the PMT. Average multiplication on the first cascade ${\nu}$ may be derived from these parameters. ${\sigma}$ parameter describes the Gaussian shape of the pedestal function. ${\xi}$ parameter describes effective cascade multiplication on the second dynode. The combination of 9 parameters describes a single-anode PMT SPE response in an ideal measurement setup with a Gaussian pedestal function. If the pedestal amplitude distribution is not exactly following the Gaussian shape the problem of parameterizing the SPE distribution requires addition of new parameters taking into account the distortion of the pedestal Gaussian. The method was successfully implemented in ~\cite{DEGTIARENKO20171} in the case of small exponential noise contribution to the Gaussian measurement function, see the Eqs. (38-40) in that publication. In the present study we attempt to use similar approach to parameterize and approximate the contribution of the crosstalk signals coming from the neighboring pixels to the SPE amplitude distribution. The model for the process, in agreement with the observations presented in the previous chapter, assumes that a portion of the signal from a neighboring pixel may be randomly added to the amplitude measured in a given pixel under investigation. Such random contribution could in principle depend on the neighbor. It would be impossible to characterize all possible pair combinations. The ad hoc approximation implemented in this work assumes that the spectral shape of the crosstalk contribution is relatively small compared with the average SPE amplitude, and can be averaged over all neighbors and parameterized using three parameters characterizing generally Poissonian shape of the crosstalk spectrum from an average "neighbor event" when the photoelectron signal in the neighboring pixel generates it. Such addition may be considered an addition to the Gaussian measurement function, similar to how it was  


 
We started by using a model for signal amplitude distribution based off of Gaussian single photoelectron spectra as discussed in \cite{Bellamy:1994bv} that works reasonably well for H8500, but this model does not satisfactorily treat the spectra seen in the new H12700 MAPMTs from Hamamatsu. Consequently, Pavel Degtiarenko~\cite{DEGTIARENKO20171} has developed a more complicated model using several Poisson components to better fit the MAPMT spectra, especially in the single photoelectron cases.
This mathematical model features a realistic description of the MAPMT response where each parameter corresponds to the physical process inside the MAPMT.
The SPE spectrum is fitted with a function used to describe the signal amplitude distribution measured by the MAPMT as shown on Fig.~\ref{fig:SPEfit},
The probability of an initial photon to knock out a photoelectron is distributed according to the Poissonian $P(m;\mu)=\frac{\mu^me^{-\mu}}{m!}$.
To approximate the performance of the first amplification cascade of the MAPMT the function $T(n,m;t)$ is introduced in the model as trinomial sum of three Poissonians with different average secondary multiplicities and the corresponding three relative probabilities for every photoelectron to generate secondary electrons.
The function $G(a,n;\sigma)$ corresponds to the realistic DAQ measurement function to introduce the experimental resolution into the resulting model function.

\begin{figure}[bt]
	\centering
	\includegraphics[width=\linewidth]{figures/SPEfit.pdf}
	\caption{Sample of single photoelectron spectrum from one of the pixels at 1000 V with low intensity laser light source,
where integer $m$ corresponds to the number of photoelectrons created at the first stage of the photodetector (photocathode) by the incident light during one event of radiation, index $n$ corresponds to the number of electrons generated at the second stage of the photodetector (first dynode).
}
\label{fig:SPEfit}
\end{figure}

The model is found to describe well the amplitude distributions measured at different levels of radiation with different supply voltages.
The parameters provide MAPMT characteristics independently of the test measurement conditions (see Fig.~\ref{fig:PavelPassport}): the $scale$ parameter is virtually independent on the light radiation level while strongly dependent on high voltage supply, the exact behavior one would expect from the characteristic of internal dynode system of MAPMT.

\begin{figure}[t]
	\centering
	\includegraphics[width=\linewidth]{figures/PavelPassport.pdf}
	\caption{Distributions of fit parameters among the pixels for measurements with 4 different high voltage supplies and 3 different light intensities: the parameter $scale$ characterizes the amplification (dynode) system, $\nu$ - first dynode performance, $\mu$ relates to the quantum efficiency (photocathode performance).}
	\label{fig:PavelPassport}
\end{figure}

Currently we have tested 80 H8500 and 260 H12700 (the largest collection of these new MAPMTs in the world).
The accumulated data provide an immense knowledge about quantum and collection efficiencies of MAPMTs, their surface uniformity, single photoelectron (SPE) spectrum resolution etc.
This parameterized information extracted from the fit of each pixel for every MAPMT is used to describe the detector response in the future simulation.


\begin{figure}[t]
	\centering
	\includegraphics[width=\linewidth]{figures/model.pdf}
	\caption{}
	\label{fig:Model}
\end{figure}