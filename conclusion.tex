\section{Conclusion}
We have tested several hundred Hamamatsu H12700 and H8500 multi-anode photomultiplier tubes in order to evaluate the application of these MAPMTs in the Ring Imaging Cherenkov detectors. The PMT’s performance study was done using the laser stand with low light illumination to simulate one-photoelectron regime as expected in the real experiment. We were using the state-of-the-art mathematical model for the characterization of the photomultipliers that were design to describe the PMT’s signals with low number of photoelectrons. 
The model was modified to take into account the effects of the cross-talk between the neighboring pixels. 
The stability of the fit parameters was proven using measurements in different laser beam intensities and at different applied high voltages. Thus, the model allowed us to characterize the photomultipliers independently of the test measurement conditions.
We carefully studied the critical for the experiments PMT's properties for the H12700 and H8500 Hamamatsu models:
the shapes of the one photoelectron spectra,
PMT's gain as a function of high voltage,
efficiency to detect one-photoelectron signal,
amplification of the first dynode that is critical for the PMT efficiency working in the one-photoelectron regime,
cross-talk between different pixels and
quantum efficiency. All these parameters were extracted for each PMT pixel.

The large amount of tested  PMT’s gives us the possibility to do statistical analysis of all parameters in hands and compare the performance of H12700 and H8500 for an application in Cherenkov detectors. The parameter database accumulated as the result of this work will be used for the selection of the MAPMTs for installation in the RICH detector, and for the optimization of the future run parameters, such as the tube placement selection, setting the values of operating high voltage, gains and thresholds of the front-end electronics. 

This data also makes it possible to estimate the spread of such parameters in the mass production of MAPMT devices as channel gain, quantum efficiency, SPE spectral shapes, crosstalk parameters between the pixels of one PMT and throughout the detector.
The results show that the quality of MAPMT mass production at Hamamatsu is high and satisfies our needs in the good quality single photoelectron detection.
 


