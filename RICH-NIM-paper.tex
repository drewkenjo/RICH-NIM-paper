%\documentclass[5p,times,longtitle,preprint]{elsarticle}
\documentclass[5p,times,preprint]{elsarticle}

\usepackage{amsmath}
\usepackage{amssymb}
\usepackage{soul}

%\usepackage[]{lineno}
%\renewcommand\linenumberfont{\ttfamily\bfseries\tiny}
%\setlength\linenumbersep{5pt}
%\linenumbers

\usepackage{enumerate}
\usepackage{graphicx}% Include figure files
\usepackage{dcolumn}% Align table columns on decimal pion\index{\footnote{}}t
\usepackage{bm}% bold math
\usepackage{color}% bold math

\usepackage{epstopdf}
\usepackage{subcaption}
\usepackage{tcolorbox}
\usepackage{url}
 
\hyphenpenalty=1500
\exhyphenpenalty=1500
\journal{Nuclear Instruments and Methods in Physics Research A}
\begin{document}
\begin{frontmatter}
\title{Characterization of Multianode Photomultiplier Tubes for use in CLAS12 RICH Detector}

\author[A]{P.~Degtiarenko }
\author[B]{A. Kim \corref{cor1}} 
\ead{kenjo@jlab.org}
\cortext[cor1]{ Corresponding author Tel: +1 757 269 6356}
\author[A]{V. Kubarovsky }
\author[A]{B. Raydo}
\author[C]{A. Smith}



\address[A]{Jefferson Lab, Newport News, Virginia, USA}
\address[B]{University of Connecticut, Storrs, CT 0626, USA}
\address[C]{Duke University, Durham, NC 2770, USA}



\begin{abstract}
We present results of the detailed study of several hundred Multianode Photomiltiplier Tubes (MAPMT) H12700 from Hamamatsu, characterizing their response to the Cherenkov light photons in the second Ring Imaging Cherenkov detector, a part in the CLAS12 upgrade at Jefferson Lab.
Using the same dedicated front-end electronics as in the main RICH detector, the laser test setup allowed us to extract the single photoelectron responses of every pixel in the MAPMT set,
characterize its properties such as gain, quantum efficiency, signal crosstalk between neighboring pixels,
and determine the optimal signal threshold values to evaluate its efficiency to Cherenkov photons.
The total number of pixels studied was 25,536. The recently published state-of-the-art computational model, describing photon detector response functions measured in the conditions of low light, was extended to include the successful description of the crosstalk contributions to the spectra.
The database of extracted parameters will be used for the final selection and arrangement of the MAPMTs in the new RICH detector, and determine their optimal operation parameters.
The results show that the quality of H12700 MAPMTs is high, satisfying our needs in the good position-sensitive single photoelectron detectors.
\end{abstract}

\begin{keyword}
Ring Imaging Cherenkov detector \sep
Hamamatsu Multianode Photomultiplier tubes H8500 and H12700 \sep
Photon detector \sep Photomultiplier  \sep
Photoelectron  \sep  Signal amplitude spectra \sep Signal crosstalk \sep
Photon detection efficiency
\end{keyword}


\end{frontmatter}

%%%%%%%%%%%%%%%%%%%%%%%%%%%%%%%%%%%%%%%%%%%%%%%%%%%%%%%%%%%%%%%%%%%%%%%%%%
%\begin{tcolorbox}[halign=center,colback=white,colbacktitle=red!40!white,colframe=red!80!white,left=0pt,right=0pt,top=0pt,bottom=0pt,boxrule=4pt,title={\bfseries\color{black}For draft only, to be removed in final version}]
%\tableofcontents
%\end{tcolorbox}

\section{Introduction}
As part of the ongoing study of the structure of nucleons \cite{Avakian:2010ae}  in Hall B at the Thomas Jefferson National Accelerator Facility (JLab)  the CEBAF Large Acceptance Spectrometer (CLAS12) \cite{Burkert:2020akg} aims to accurately identify the secondary particles of high energy reactions, assist in probing the strangeness frontier, and aid in characterizing transverse momentum distribution (TMD) and generalized parton distribution (GPD) functions. Indispensable to this task is the ability to identify kaons, pions, and protons.  With the CLAS12 spectrometer providing accurate momentum measurements the Ring Imaging Cherenkov detector (RICH) \cite{Contalbrigo:2020,Contalbrigo:2020snw,Mirazita:2017vav,Contalbrigo:2014rqa} provides tandem Cherenkov light-cone radius measurements which yield the velocities of near light-speed particles, thus facilitating mass-dependent particle identification.

\begin{figure}[h!bt]
	\centering
	\includegraphics[width=0.65\linewidth]{figures/RICHdetector.pdf}
	\includegraphics[width=0.65\linewidth]{figures/RICHpanel_front.png}
	\caption{Top: CLAS12 detector with RICH detector covering one out of six sectors. Bottom: the photo-matrix of multianode photomultipliers and mirror system.}
	\label{fig:RICHdetector}
\end{figure}

The photomatrix  wall is a crucial component of the RICH detector (see Fig.~\ref{fig:RICHdetector}). It is relatively large (area about 1 m$^2$) and should be comprised of many photon detection devices such as photomultiplier tubes.
Due to the imaging aspect of the RICH they must provide a spatial resolution of less than 1 cm.
Since multiple photon detectors are tiled into large arrays, they should have large active area with minimal dead-space.
The photon detectors must also efficiently detect single photon level signals and should be sensitive to the visible light due to the aerogel radiator material.
MultiAnode PhotoMultiplier Tubes (MAPMTs) are perfect candidates for the CLAS12 RICH detector.
They are the flat-panel Hamamatsu MAPMTs offering an adequate compromise between detector performance and cost.
Each MAPMT comprises an 8 by 8 array of pixels, each with dimension of 6 by 6 mm.
Furthermore, the device has a very high packing fraction of 89\% with a high quantum efficiency in the visible light region.
The tubes also have excellent immunity to magnetic fields, because all internal parts are housed in a metal package and the distance between dynode electrodes are very short.


Initially, the Hamamatsu H8500 MAPMT model \cite{H8500} was chosen as the best option because they provide high quantum efficiency for visible light and sufficient spatial resolution (6x6 mm$^2$) at a limited cost.  However, later Hamamatsu has released the new H12700 MAPMT model  \cite{H12700} which shows enhanced single photoelectron (SPE) detection, reduced cross-talk between pixels and is otherwise similar to the H8500 MAPMTs in spatial resolution and  cost. The first RICH detector has been installed in sector 4 of the CLAS12 detector in 2018. There are 391 Hamamatsu MAPMT  in the photodector matrix, 76 of them are H8500 model and 315 H12700 model. 
The initial characterization of these PMTs was done using laser stand equipped by the crate of Flash ADC  boards \cite{FADC250} because the custom RICH front-end electronics \cite{RICH_FE} was not ready by that time. The second RICH detector is under construction presently and is planned to be installed to the sector 1 of CLAS12 in 2022. The second detector is almost identical to the first one. The characterization of PMTs for this detector was done with standard front-end boards that have much better parameters than FADC used in the past. We made detailed characterization around 400 H12700 PMTs as well as several H8500 model to make the comparison of two models. These data give us the possibility to better understand the performance of the first RICH where we are using both PMT models. The results of this study is presented in this paper.



	  % vpk

\section{Laser stand for the MAPMT characterization}
The large number of the channels in the RICH detector  poses challenging problem for the MAPMT testing and calibration.
RICH consists of 391 MAPMTs resulting in total number of channels equal to 25024. So in order to test them efficiently within a reasonable timeframe the fully automated test stand was build to evaluate 6 MAPMTs at once, as shown on Fig.~\ref{fig:MAPMTtest}.

\begin{figure}[hbt]
	\centering
	\includegraphics[width=0.9\linewidth]{figures/blackbox.png}
	\caption{Inner view of the laser stand.}
	\label{fig:MAPMTtest}
\end{figure}

The test stand consists of picosecond diode  laser PiL047X with 470 nm wavelength, 2 long travel motorized stands to drive laser fiber in two dimensional space for individual pixel illumination, the motorized wheel with neutral density filter system, 2 adapter boards for MAPMT with JLab designed front-end electronics boards \cite{}{Contalbrigo:2020}.
The laser light is directed through the fiber and attenuated to the single photon level using the neutral density filters to mimic the conditions of the RICH detector.
The motors were remotely controlled to move the focused laser beam across (see Fig.~\ref{fig:beamopt1}) the entire surface of the MAPMT entrance window and illuminate one by one of all its 64 pixels individually.
Another option is to illuminate the whole surface of MAPMT photocathode at once using the Engineered Diffuser to produce square pattern with non-Gaussian intensity distribution (see Fig.~\ref{fig:beamopt2}). 

All laser stand equipment is sitting in the black box with non-reflective black material on the optical table. The laser interlock safety box automatically switch off laser, as well as front-end low voltage  electronics and MAPMT high voltage to prevent the possible photomultiplier damage or laser light human illumination in case if somebody will try to open the front door of the black box during the measurements.

\begin{figure}[bt]
	\centering
	\begin{subfigure}[b]{0.628\linewidth}
		\includegraphics[width=\linewidth]{figures/beamspot.pdf}
		\caption{Focused laser beam with the dimension much less than the  MAPMT pixel size.}
		\label{fig:beamopt1}
	\end{subfigure}
	\begin{subfigure}[b]{0.354\linewidth}
		\includegraphics[width=\linewidth]{figures/beamsquare.pdf}
		\caption{Square pattern illuminated all MAPMT surface.}
		\label{fig:beamopt2}
	\end{subfigure}
	\caption{The laser light output options.}
\end{figure}

This configuration brings routine workload to minimum allowing the evaluation of 6 MAPMTs (equivalent to 328 conventional PMTs!) at 4 different high voltages and 6 different light intensities within 6 hours with less than 15 minutes of human intervention needed to load the MAPMTs to the front-end boards.

\begin{figure}[b]
	\centering
	\begin{subfigure}{0.3\linewidth}
		\includegraphics[width=\linewidth]{figures/surfaceuniform1.pdf}
		\caption{MAPMT with visible internal structure of metal channel dynodes and focusing mesh.}
		\label{fig:surfaceuniform1}
	\end{subfigure}
	\quad
	\begin{subfigure}{0.3\linewidth}
		\includegraphics[width=\linewidth]{figures/surfaceuniform3.pdf}
		\caption{The response along the X axis; the signal drops in the deadspace between the pixels.}
		\label{fig:surfaceuniform2}
	\end{subfigure}
	\quad
	\begin{subfigure}{0.3\linewidth}
		\includegraphics[width=\linewidth]{figures/surfaceuniform2.pdf}
		\caption{The response along the Y axis: multiple segmentations within the pixels.}
		\label{fig:surfaceuniform3}
	\end{subfigure}
	\caption{The response uniformity of MAPMT.}
	\label{fig:surfaceuniform}
\end{figure}


Before starting the systematic study of the MAPMT responses, a finer two dimensional scan of several pixels was performed in order to verify the uniformity of the response across pixel's surfaces, as shown on Fig.~\ref{fig:surfaceuniform}.
The horizontal and vertical axes denote laser beam position during the scan.
Along the both directions there are obvious drops in efficiency when the laser strikes the space between the pixels.
The drops are relatively narrow so the dead-space is very small as expected from the Hamamatsu specifications.
Additionally, a vertical efficiency variation is visible across the pixel in horizontal scan.
These inhomogeneities are correlated with the vertical walls separating dynode chains, owing to the constructional features of the MAPMT. The discontinuity in dynode structure is visible  on Fig.~\ref{fig:surfaceuniform1}.
The separate response maps for photocathode shows relatively uniform signal without efficiency drops, confirming that the variation arises from the dynode system.

\section*{Front-end electronics}

The highly integrated front-end electronics with modular design was developed for a large array of MAPMT H12700 to minimize the impact of the electronics material on the detector downstream the RICH.
An architecture of the readout electronics consists of front-end cards with dedicated Application Specific Integrated Circuit (ASIC) configured, controlled and readout by programmable devices such as Field Programmable Gate Array (FPGA).
The ASIC board is based on the MAROC3 integrated circuit whose excellent single photon capabilities both in analog and binary mode have been confirmed.
The final design has consists of stacked PCB layers behind each MAPMT sensor (see~Fig.~\ref{fig:feboards}).
The first layer houses the ASIC front end and ancillary components (e.g. external amplifier) and it is directly connected to the anodes array.
A second PCB will host the FPGA in charge of configuring, managing and acquiring one or more ASICs and the low voltage and HV bias distribution.
The use of the JLab SSP as controller and collector of the front-end data provides a strong synergy with the current JLab upgrade activity.
Data are transmitted on high speed serial (optical) lines minimizing the wiring and therefore the material budget.
With that sandwich architecture the total photon detection surface will be covered by a fixed number of basic units or tiles made up by two or three sensor each.
The total spacing for electronics will not exceed 20 cm in depth (including MAPMT and mechanical support).
The three-tiles electronics module with and without 3 H12700 MAPMTs installed is shown on~Fig.~\ref{fig:feboards}.

\begin{figure}[htb]
  \centering
  \includegraphics[width=0.9\linewidth]{figures/fe1.pdf}
  \includegraphics[width=0.9\linewidth]{figures/frontendPMT.pdf}
  \caption{Front-end electronics readout board and mounted MAPMTs.}
  \label{fig:feboards}
\end{figure}

The MAROC chip consists of 64 independent channels.
The single channel has a pre-amplification and configurable 8 bit gain correction stage followed by independent binary and analog lines.
The binary line is digital and most suitable for the RICH application.
We are planning to analyze the performance of MAROC chip as a function of its threshold and gain.
Additionally front-end electronics stability and noise will be tested as well.


\section*{Front-End Electronics for RICH}


Due to the high number of active channels in readout subsystem of the RICH detector and available limited space the Multi Anode ReadOut Chips (MAROC) were chosen to construct highly integrated front-end electronics with modular design.
The MAROC performances were checked and found suitable for the RICH requirements.
Each chip is equipped with single channel adjustable preamplifier, configurable signal shaping, slow shaper capable of charge measurements and binary output with fast shaping and adjustable threshold.
The custom modular design as shown on Fig.~\ref{fig:FEpics} consists of sandwich architecture where one board hosts the ASIC chips (2 or 3 per board) and another active board hosts the FPGA to manage and configure the ASICs, the third board is a passive adapter for MAPMT sockets.
It satisfies the following requirements:
\begin{itemize}
\item 100\% efficiency at 1/3 of single photoelectron signal (50 fC)
\item time resolution of 1 ns
\item short deadtime to sustain the trigger rate of 20 kHz
\item latency of 8 $\mu s$
\end{itemize}
The conceptual design of the electronics readout boards was finished in the RICH development phase and currently we have several board implementations available at Jefferson Lab and INFN for final tests.

The preliminary measurements with internal onboard charge injector, external charge injector and signal generator were performed to test and calibrate FE electronics.
However, the measurements of custom front-end electronics together with installed MAPMTs in the RICH Black Box setup are crucial for understanding the their future performance in the RICH detector in CLAS12.
RICH MAPMT test setup was modified to house two FE board at once inside the black box as shown on Fig.~\ref{fig:FEmount}.
The focus of the modification was to adapt the test setup in such way that the swap of FE boards would be fast and easy.
The PCB guidelines were installed inside the black box to ensure easy mounting and dismounting procedure universal for both 2-MAPMT and 3-MAPMT FE modules.
This requirement exists in light of the future measurements that our group is expected to perform on all FE modules for testing purposes.
Each FE module is connected to the low voltage power supply to power FPGA and ASIC boards. The communication between FPGA board and PC is performed using TCP/IP protocol via optical fiber network cable.
The HV cables, one per each module, supply power for attached MAPMTs.
The Data Acquision program runs on external PC under Linux OS, configures FPGA and MAROC boards and collects the data through the network interface.
As before the laser with neutral density filters and light diffuser is installed on moving platform to allow illumination of individual MAPMTs with different light intensities.
The current setup allows fast evaluation of FE modules with highly automatized procedure which is important because RICH will be utilizing 113 tiles with 3 MAPMTs and 23 tiles with 2 MAPMTs FE modules to house 391 MAPMTs.

The planned measurements will group individual boards with MAPMTs and treat them as a inseparable unit that will go into the final assembly of RICH detector.
Every unit will be tested in the black box setup to mimic the conditions of SPE regime expected during the RICH routine operations.
The calibration procedures will be developed and evaluated during these test measurements.
The gain equalization, individual channel amplification parameters and optimal threshold values will be investigated and resolved.
In addition, self-triggering capabilities of FE modules allow careful dark current measurements.
The previous measurements and evaluation of MAPMTs with different readout systems (JLab FADC) will be compared with FE measurements to better understand the MAPMTs and FE performances.
The calibration procedures will be developed and evaluated during these test measurements.

\begin{figure}[hbt]
  \centering
  \includegraphics[width=0.9\linewidth]{figures/LaserSetup.jpg}
  \caption{Mount inside RICH Black box is capable to hold two Front-End boards and up to 6 MAPMTs}
  \label{fig:FEmount}
\end{figure}

The MAROC chip used to read MAPMT outputs has two main paths to feed the preamplified input current:
\begin{itemize}
\item analog line with a slow shaper which allow the injected charge measurements
\item binary line with a fast shaper followed by a discriminator with configurable threshold which allows to deliver trigger outputs
\end{itemize}
Both lines are independent and highly configurable.
The slow shaper can not be used during normal operation of RICH detector due to relatively small HOLD latency of 200 ns.
It can, however, be used for calibration purposes.
On the other hand the digital line with fast shaper is suitable for RICH operation and will be used to detect signals from Cherenkov photons incident on the readout panel.

The preliminary measurements of MAPMTs signal were taken using the FE tile with 3-MAPMTs installed inside the RICH black box.
Both lines were evaluated as shown of Fig.~\ref{fig:slowWaveform} and~\ref{fig:fastWaveform}
The internal pulse generator was used to trigger laser through its adjustable external trigger input achieving synchronization between light source and MAROC readout.
Then the data were collected with both lines collecting data in parallel.
Fig.~\ref{fig:slowMAROC} shows the slow shaper waveform for different charge injections from MAROC3 datasheet.
On the right Fig.~\ref{fig:slowJLab} shows the same waveform obtained from the data collected at Jefferson Lab using RICH test setup with laser and MAPMT.
In order to reconstruct the waveform of the slow shaper, the hold delays parameters was changed from 1 to 100 and data were collected at different values.
This measurements alow us to find the best value of the hold delay for internal ADC conversions which correpond to the signal maximum at around (13 ticks = 104 ns).
It ensures the largest precision to the charge measurements.
The shape of the analog output waveform is configurable and its gain and peaking speed can be changed.


\begin{figure}[hbt]
	\centering
	\begin{subfigure}{0.49\linewidth}
		\includegraphics[width=\linewidth]{figures/SlowWaveForm_MAROC.pdf}
		\caption{Slow shaper waveform from MAROC datasheet for different charge injections}
		\label{fig:slowMAROC}
	\end{subfigure}
	\begin{subfigure}{0.49\linewidth}
		\includegraphics[width=\linewidth]{figures/SlowWaveForm_RICH.pdf}
		\caption{Slow shaper waveform from JLab data for MAPMT signal}
		\label{fig:slowJLab}
	\end{subfigure}
	\caption{Slow shaper response from MAROC}
	\label{fig:slowWaveform}
\end{figure}



\begin{figure}[hbt]
	\centering
	\begin{subfigure}{0.9\linewidth}
		\includegraphics[width=\linewidth]{figures/FastShaper_MAROC.pdf}
		\caption{Fast shaper waveform from MAROC datasheet}
		\label{fig:fastMAROC}
	\end{subfigure}
	\begin{subfigure}{0.6\linewidth}
		\includegraphics[width=\linewidth]{figures/FastWaveForm_RICH.pdf}
		\caption{Measured dependency of "time over threshold" vs threshold}
		\label{fig:fastJLab}
	\end{subfigure}
	\caption{Fast shaper response from MAROC}
	\label{fig:fastWaveform}
\end{figure}


The digital line can be sampled with a predefined clock on an adequate deep external digital pipeline with event information available promtly in parallel making it suitable for the RICH application.
The binary information comes from the discrimination of the fast shaper lines output.
On Fig.~\ref{fig:fastMAROC} the fast shaper waveform is shown for fixed injected charge from MAROC3 datasheet.
On the right fig.~\ref{fig:fastJLab} shows the the dependence of time over threshold versus threshold values.
The digital line reports the time when fast shaper waveform crosses the thhreshold value giving us the leading and trailing edges.
To reconstruct fast shaper waverform the measurements with different threshold values were collected and the time of leading and trailing edges were plotted.
As expected at higher thresholds the signal crosses the threhold later and has a shorter time over threshold.
Also the spread of SPE response from MAPMT gives us preliminary indication of signal's time walk which can be further corrected to improve time resolution.
The behaviour at high threshold might not represent the expected performance because the signal reaches saturation and loses its linearity.


Preliminary measurements of MAROC3 and MAPMTs demonstrate good time and charge resolution allowing to further improve our MAPMTs measurements and their understanding.
The final production of FE boards in nearly complete and we expect to start receiving FE tiles within the next month.
  	  % Andrey
The measurements of custom front-end electronics together with the installed MaPMTs in the RICH black box setup were crucial to understand their performance in the RICH detector.
To test and calibrate it, multiple tests with an internal onboard charge injector, an external charge injector, and a signal generator were performed.
As shown in Fig.~\ref{fig:MAPMTtest}, the RICH MaPMT test setup can house two FE boards inside the black box.
The communication between the FPGA board and the PC is performed using TCP/IP protocol over optical Ethernet (1000BASE-SX).
The data acquisition program executes on a remote PC running Linux OS, configures the FPGA and MAROC boards, and collects the data through a network interface.
The current setup allows fast evaluation of the FE modules with a highly automated procedure, which is important because the RICH panel consists of 115 tiles with 3-MaPMT and 23 tiles with 2-MaPMT FE modules.
  	  % Andrey
\section{MAROC chip calibration}

To remove the non-linearity in the ADC readout, a procedure was developed to convert the amplitude of the MAROC slow shaper signal from ADC channels into charge. The MAROC has a built-in charge injection functionality consisting of a test input pin that is connected to the preamplifiers through a logic network of switches and 2 pF capacitors.
Together with an external step function generator, this can be used to inject a controllable amount of charge directly into the preamplifiers. We measured the output of the slow shaper in ADC channels for 82 different input charges ranging from 0 to 4 pC.
Figure~\ref{fig:MAROCcalib} shows the relationship between the injected charge and the measured amplitude in units of ADC channels for three different readout channels. The relationship between charge and ADC channels is linear up to about 1.5 pC.
This distribution was observed to vary between chips and pixels, and thus individual distributions were measured for all 64 pixels on each MAROC used in this study. 

\begin{figure}[hbt]
	\centering
	\includegraphics[width=\linewidth]{figures/adc_v_charge.png}
	\caption{Response of the MAROC slow shaper in ADC channels as a function of the injected charge. The curves shown are for pixel \#1 in three different MAROC boards.}
	\label{fig:MAROCcalib}
\end{figure}

This calibration data was used to convert the measured amplitude in ADC channels into charge collected on an event-by-event basis. A local polynomial regression was used to provide a one-to-one mapping of adc channel to charge. Figure ~\ref{fig:H12700calib} and Fig.~\ref{fig:H8500calib} show typical amplitude distributions before and after this conversion was applied for one H12700 MAPMT pixel and one H8500 MAPMT pixel, respectively. For both, the conversion to charge extends the high-amplitude tails of the spectra due to the non-linearity of the ADC readout.

\begin{figure}[hbt!]
	\centering
	\includegraphics[width=\linewidth]{figures/GA0982_w1_g064_v1100_063_adc_charge.png}
	\caption{Top: A typical SPE spectrum for one H12700 pixel in units of ADC channel. Bottom: The same spectrum after converting the units into pC.}
	\label{fig:H12700calib}
\end{figure}

\begin{figure}[hbt!]
	\centering
	\includegraphics[width=\linewidth]{figures/CA7811_w1_g064_v1100_063_adc_charge.png}
	\caption{Top: A typical SPE spectrum for one H8500 pixel in units of ADC channels. Bottom: The same spectrum after converting the units into pC.}
	\label{fig:H8500calib}
\end{figure}
 % Drew
\section{Cross talk measutrements}

Electronic crosstalk is present in both the H12700 and H8500 MAPMTs. This can be observed by plotting the measured charge in one pixel vs. the measured charge in an adjacent pixel. Fig.~\ref{fig:H12700neighbors} and Fig.~\ref{fig:H8500neighbors} show these two dimensional plots for all pixels which neighbor pixel 28 for one H12700 MAPMT and one H8500 MAPMT, respectively. The crosstalk bands are most prominently seen in the pixels directly to the left or right, where the amount of crosstalk is shown to be proportional to the charge measured in the central pixel. 


\begin{figure*}
	\centering
	\includegraphics[width=0.95\linewidth]{figures/GA0982_neighbors_crosstalk.png}
	\caption{}
	\label{fig:H12700neighbors}
\end{figure*}
\begin{figure*}
	\includegraphics[width=0.95\linewidth]{figures/CA7709_neighbors_crosstalk.png}
	\caption{}
	\label{fig:H8500neighbors}
\end{figure*}


\subsection{Offline Crosstalk Removal}

A simple method was developed to remove the crosstalk offline on an event-by-event basis. Crosstalk events in neighboring pixels are characterized by bands in the two dimensional plots showing the measured charge in one pixel against the maximum measured charge in the neighboring pixels. In cases where light only entered the central pixel, the neighboring pixels may have some non-zero signal which should be proportional to the amount of charge collected in the central pixel. This can be seen in the example shown below which plots the maximum charge measured in the neighboring pixels of 
		  % Drew
\section{Calibration of laser photon flux}

The calibration of the absolute laser photon flux was performed with the use of the silicon photodiode Hamamatsu S2281.
The tabulated quantum efficiency of this diode at the wavelength of our laser ($\lambda=470$ nm) is 62.6\%. 
The active part of the diode is a circle with a diameter of 11.3~mm, which is 100~mm$^2$. 
A KEITHLEY 6485 picoammeter was used to measure the average diode current while illuminated by the laser beam.
The noise diode current was estimated to be at the level of 0.2~pA. 
During the MAPMT characterization, the laser frequency was maintained at 20 kHz. 
For light calibration, the higher the frequency, the better the current measurement accuracy that can be achieved from the point of view of the noise level. 
The maximum frequency of our laser is 1 MHz.
However, there are additional systematic uncertainties associated with the extrapolation from one frequency to another. 
For this reason, the scan of the light field was done at a working frequency of 20 kHz. 
The measured current in the center position of the laser head was around 29.2~pA at this frequency, meaning the systematic uncertainty
of this measurement was below 1\%.  We made a detailed two dimensional scan of the photon flux by
moving the laser head with step sizes of 2~mm in the X and Y directions along the full area where the 3 MAPMTs were located during the characterization procedure.
Normalized to one laser pulse and 1~mm$^2$ area, the number of photons with $\lambda=470$~nm  is presented in Fig.~\ref{fig:light_flux}.
The maximum value of the photon flux in the center of the light field equals 145 $\gamma/mm^2/pulse$.
These measurements were done without  any optical filters installed. We used neutral density calibrated optical filters with anti-reflection coating.
To check the possible filter effects we made a measurement of the light flux for one of the filters with a tabulated attenuation of 100. 
This test was done with a frequency of 1 MHz to increase the accuracy of the current measurement. 
The ratio of the measured attenuation factor to that tabulated was determined to be 1.05$\pm 0.01$. This coefficient was applied to the map of the photon flux when used for data with optical filters. It takes into account the possible effects of rescattering or reflection of the photons by the filters.
\begin{figure}[h]
\centering
\includegraphics[width=0.5\textwidth]{figures/photon_flux.pdf}
\caption{Number of photons per mm$^2$ in  one laser pulse.}
\label{fig:light_flux}
\end{figure}

The knowledge of the absolute number of photons hitting the photomultiplier tubes during the characterization gave us the possibility to measure the quantum efficiency of the MAPMTs for each pixel. The average number of photoelectrons, $\mu$, is proportional to the quantum efficiency:
$$
\mu=\epsilon_{QE} \int_{S_{pixel}}\frac {dN_\gamma}{dS} dS,
$$
\noindent
where $\int_{S_{pixel}}\frac {dN_\gamma}{dS} dS$ is the number of photons integrated over the pixel's area, $S_{pixel}$, and $\epsilon_{QE}$ is the quantum efficiency of the pixel.
The integration included the measured light field at the position of the pixel under study.
The parameter $\mu$ was determined during the PMT characterization. % Drew
\section{Mathematical model for the description of the PMT response}

The goal for using MAPMTs in RICH detectors is to achieve reliable detection of single photons in the Cherenkov light radiation cones. Single photon incident on a PMT face may knock out a single photoelectron from the PMT's photocathode with a certain probability, defined as Quantum Efficiency (QE). The photoelectrons cascade inside the PMT to generate a typical amplified electrical signal at the anode. The amplitude distribution of the single photoelectron (SPE) signals depends on the MAPMT design and high voltage applied, and varies from pixel to pixel. Tests and characterization of multiple MAPMTs include measuring the SPE amplitude distributions for every pixel, finding out the appropriate amplitude thresholds, and determining QE. To achieve this goal we used the methods developed in ~\cite{DEGTIARENKO20171}, expanded to include the new empirical method to take into account the effects of the pixel-to-pixel cross talk in H12700 tubes. The reference ~\cite{DEGTIARENKO20171} describes in detail the mathematical model used to extract and parameterize the SPE distributions from the measurements using the laser test setup. The method allows in principle to describe the SPE functions of essentially any complexity by decomposing them into a sum of Poisson distributions with different averages. For the detailed explanations and the definition of the model parameters please refer to ~\cite{DEGTIARENKO20171}. The list of main parameters includes $\mu$, average number of photoelectrons produced by the laser in a given pixel per one test pulse, and {\it{scale}}, average amplitude of the SPE distribution in pC. Five parameters determine the shape of the SPE distribution, defined as a normalized sum of three Poisson distributions with different average multiplication coefficients applied to the photoelectron on the first dynode of the PMT. Average multiplication on the first cascade ${\nu}$ may be derived from these parameters. ${\sigma}$ parameter describes the Gaussian shape of the pedestal function. ${\xi}$ parameter describes effective cascade multiplication on the second dynode. The combination of 9 parameters describes a single-anode PMT SPE response in an ideal measurement setup with a Gaussian pedestal function. If the pedestal amplitude distribution is not exactly following the Gaussian shape the problem of parameterizing the SPE distribution requires addition of new parameters taking into account the distortion of the pedestal Gaussian. The method was successfully implemented in ~\cite{DEGTIARENKO20171} in the case of small exponential noise contribution to the Gaussian measurement function, see the Eqs. (38-40) in that publication. In the present work we use similar ad hoc approach to parameterize and approximate the contribution of the cross talk signals coming from the neighboring pixels to the SPE amplitude distribution. The model for the process, in agreement with the observations presented in the previous chapter, assumes that a portion of the signal from a neighboring pixel may be randomly added to the amplitude measured in a given pixel under investigation. Such random contribution could in principle depend on the neighbor. It would be impossible to characterize all possible pair combinations. Generally Poissonian shapes of the SPE distribution functions suggests the shape of the cross talk contribution in the form similar to a Poisson distribution scaled to represent the portion of the charge generated in the neighboring pixel and transferred to the pixel studied. A reasonably good approximation of such scaled continuous Poisson-like distribution may take the form 
\begin{equation}
\label{ContPoisson}
 P_c(x;\lambda) = \frac{\lambda^{x} e^{-\lambda}}{\Gamma(x+1)},
\end{equation}
similar to the original Poisson distribution defined for integer x, but expanding the definition to all real $x>0$ values, using the fact that for integer $x = n$ gamma function $\Gamma(x+1) = n!$, and it is continuous in $x$. The function $ P_c(x;\lambda) $ has its average value at $\lambda$ similar to the standard Poisson distribution, and is approximately normalized as a probability distribution. The normalization holds very well at reasonably large $\lambda > \approx 2$, and can be improved in the calculation algorithm. The function of Eq. (\ref{ContPoisson}) may serve as a probability distribution function describing the contribution of an averaged cross talk electron to the SPE distribution
\begin{equation}
\label{fe}
 f_{e}(a;\zeta,\lambda) = \frac{\lambda}{\zeta} P_c \left (\frac{a \lambda}{\zeta};\lambda \right ),
\end{equation}
where $a$ is normalized signal amplitude; $\zeta$ is the scale factor characterizing the average amplitude of an average cross talk signal in units of the $scale$ parameter; and $\lambda$ is the parameter responsible for the Poissonian shape of the cross talk distribution. $\zeta$ must be small compared with $scale$ for the model to work reliably, which is the case for H12700 MAPMTs. Parameter $\lambda$ accommodates the shape of the cross talk function and reflects the effective separation of the cross talk amplitude peak from the pedestal.

One more parameter required for the cross talk model description is $\beta$, which represents the average number of the cross talk electrons per one laser pulse, and is expected to be comparable with $\mu$. Similar to how the measured amplitude distribution during the test is defined by the Poisson sum on the number of photoelectrons $m$ with average $\mu$, the measured cross talk amplitude distribution is defined by the Poisson sum on the number of the cross talk electrons in the neighboring pixels, with average $\beta$: 
\begin{equation}
\label{fCT}
 f_{CT}(a;\beta,\zeta,\lambda) = \sum\limits_{i=1}^{\infty}  \frac{\beta^{i} e^{-\beta}}{i!} \left [ f_{e}(a;\zeta,\lambda) \right ] ^{*i},
\end{equation}
assuming also that similarly to the integer Poisson distributions, the convolution powers of the electron cross talk functions $f_{e}$ may be explicitly calculated as  
\begin{equation}
\label{fe_cp}
 \left [ f_{e}(a;\zeta,\lambda) \right ] ^{*i} = \frac{\lambda}{\zeta} P_c \left (\frac{a \lambda}{\zeta};i\lambda \right ).
\end{equation}

The approximation implemented in this work assumes that the amplitude of the cross talk contribution is relatively small compared with the average SPE amplitude and may be considered an addition to the Gaussian measurement function, similar to how it was implemented in Eq.~(39) of ~\cite{DEGTIARENKO20171}. The Gaussian form $G(a,n;\sigma_{\mathrm{eff}})$ in the main model calculation Eq.~(36) of ~\cite{DEGTIARENKO20171} can be substituted with the corrected measurement function  
\begin{equation}
  \label{EMGform}
\exp(-\beta) \times G(a,n;\sigma_{\mathrm{eff}}) + f_{CT}(a;\beta,\zeta,\lambda) .
\end{equation}

The technique is illustrated in Fig.~\ref{fig:Model} showing an example of an SPE function for the distribution of the test events on the normalized measured charge $a$, with $a=1$ corresponding to the average charge collected from one photoelectron. The series of lines marked as $m = 1, 2, 3$ correspond to the charge distributions in the events with corresponding number of photoelectrons, assuming the average number of photoelectrons in test event is $\mu = 0.2$. Red distribution corresponds to the pedestal measurement function $G$ with the added cross talk correction. Regions in this distribution marked with $\mathrm{N_{cte}=0, 1, 2}$ correspond to the original Gaussian pedestal function and the contributions from one and two cross talk electrons. The parameters are selected for better visibility of the cross talk effects, with $\beta$ equal to $\mu$, $\zeta$ equal to 10\% of the $scale$ parameter, and $\lambda = 5$ to make the cross talk Poisson peak more visible.  

The fitting procedure from ~\cite{DEGTIARENKO20171} was modified to include the new three parameters in the FORTRAN routine describing the measured test spectra, bringing the total number of parameters to 12. The algorithm for the multiparametric minimization was adjusted correspondingly to provide stability. The experimental verification of the fit stability and reproduceability of the results was performed using multiple measurements of the same MAPMTs in the different slots in the test setup and comparing the results. Overall confidence was also assured by extracting parameters for each MAPMT in several test conditions, varying the high voltage and the illumination conditions, and verifying the consistency in the values of the extracted parameters. 

\begin{figure}[t]
	\centering
	\includegraphics[trim=90 60 110 100, clip, width=\linewidth]{figures/model.pdf}
	\caption{Model signal charge distribution (black line) illustrating the parameterization for the cross talk effects. Red trace ($m = 0$) corresponds to the pedestal measurement function with the additional cross talk contribution, violet lines ($m = 1, 2, 3$) show contributions from events with the number of photoelectrons 1, 2, and 3, with their relative probability corresponding to the Poisson distribution with average $\mu = 0.2$  }
	\label{fig:Model}
\end{figure}
		  % pavel
\section{Characterization of MAPMTs}

\begin{figure*}[hbt] 
\centering 
  \subfloat[3 mm mask]{%
    \includegraphics[clip=true,trim=100 75 140 100,width=.245\textwidth,height=.15\textwidth]
                    {figures/CA7811_a.pdf}}
  \subfloat[6 mm mask]{%
    \includegraphics[clip=true,trim=100 75 140 100,width=.245\textwidth,height=.15\textwidth]
                    {figures/CA7811_b.pdf}}
  \subfloat[No mask, cross talk removed by software]{%
    \includegraphics[clip=true,trim=100 75 140 100,width=.245\textwidth,height=.15\textwidth]
                    {figures/CA7811_c.pdf}}
  \subfloat[No mask, cross talk approximated by fit]{%
    \includegraphics[clip=true,trim=100 75 140 100,width=.245\textwidth,height=.15\textwidth]
                    {figures/CA7811_d.pdf}}
  \caption{Signal amplitude probability distributions for PMT CA7811 (H8500), pixel 9, at HV = 1000 V. a) 3mm mask. b) 6mm mask. c) run with full PMT face open, cross-talk events removed by the correlation analysis. d) run with full PMT face open, the contribution to the spectrum from the cross-talk events is approximated and parameterized by the analysis algorithm. The cross talk effects are too wide to be approximated correctly.
    }
\label{fig:CA7811}
\end{figure*}

\begin{figure*}[h!bt] 
\centering 
  \subfloat[3 mm mask]{%
    \includegraphics[clip=true,trim=100 75 140 100,width=.245\textwidth,height=.15\textwidth]
                    {figures/GA0516_1a.pdf}}
  \subfloat[6 mm mask]{%
    \includegraphics[clip=true,trim=100 75 140 100,width=.245\textwidth,height=.15\textwidth]
                    {figures/GA0516_1b.pdf}}
  \subfloat[No mask, cross talk removed by software]{%
    \includegraphics[clip=true,trim=100 75 140 100,width=.245\textwidth,height=.15\textwidth]
                    {figures/GA0516_1c.pdf}}
  \subfloat[No mask, cross talk approximated by fit]{%
    \includegraphics[clip=true,trim=100 75 140 100,width=.245\textwidth,height=.15\textwidth]
                    {figures/GA0516_1d.pdf}}
  \caption{Signal amplitude probability distributions for PMT GA0516 (H12700), pixel 4, at HV = 1000 V. a) 3mm mask. b) 6mm mask. c) run with full PMT face open, cross-talk events removed by the correlation analysis. d) run with full PMT face open, the contribution to the spectrum from the cross-talk events is approximated and parameterized by the analysis algorithm.
    }
\label{fig:GA0516_1}
\end{figure*}
\begin{figure*}[h!bt] 
\centering 
  \subfloat[3 mm mask]{%
    \includegraphics[clip=true,trim=100 75 140 100,width=.245\textwidth,height=.15\textwidth]
                    {figures/GA0516_2a.pdf}}
  \subfloat[6 mm mask]{%
    \includegraphics[clip=true,trim=100 75 140 100,width=.245\textwidth,height=.15\textwidth]
                    {figures/GA0516_2b.pdf}}
  \subfloat[No mask, cross talk removed by software]{%
    \includegraphics[clip=true,trim=100 75 140 100,width=.245\textwidth,height=.15\textwidth]
                    {figures/GA0516_2c.pdf}}
  \subfloat[No mask, cross talk approximated by fit]{%
    \includegraphics[clip=true,trim=100 75 140 100,width=.245\textwidth,height=.15\textwidth]
                    {figures/GA0516_2d.pdf}}
  \caption{Same as Fig.~\ref{fig:GA0516_1}, but all the data taken at HV = 1100 V. 
    }
\label{fig:GA0516_2}
\end{figure*}
\begin{figure*}[h!bt] 
\centering 
  \subfloat[6 mm mask, at HV = 1000 V]{%
    \includegraphics[clip=true,trim=100 75 140 100,width=.245\textwidth,height=.15\textwidth]
                    {figures/GA0516_3a.pdf}}
  \subfloat[6 mm mask, at HV = 1100 V]{%
    \includegraphics[clip=true,trim=100 75 140 100,width=.245\textwidth,height=.15\textwidth]
                    {figures/GA0516_3b.pdf}}
  \subfloat[No mask, at HV = 1000 V]{%
    \includegraphics[clip=true,trim=100 75 140 100,width=.245\textwidth,height=.15\textwidth]
                    {figures/GA0516_3c.pdf}}
  \subfloat[No mask, at HV = 1100 V]{%
    \includegraphics[clip=true,trim=100 75 140 100,width=.245\textwidth,height=.15\textwidth]
                    {figures/GA0516_3d.pdf}}
  \caption{Signal amplitude probability distributions for PMT GA0516 (H12700), pixel 4, wheel position 2, at HV = 1000 V (left plots) and at HV = 1100 V (right plots). (a) and (b) run with 6 mm mask covering the full PMT face except pixel 4. (c) and (d) run with full PMT face open, the contribution to the spectrum from the crosstalk events is approximated and parametrized by the analysis algorithm. Contributions to the spectra are shown by colors: pink is the single photoelectron, violet - two or more photoelectrons, green-black dashed line shows the measurement function including the pedestal Gaussian and the cross talk contribution. 
    }
\label{fig:GA0516_3}
\end{figure*}
\begin{figure*}[h!bt] 
\centering 
  \subfloat[6 mm mask, at HV = 1000 V]{%
    \includegraphics[clip=true,trim=100 75 140 100,width=.245\textwidth,height=.15\textwidth]
                    {figures/GA0516_4a.pdf}}
  \subfloat[6 mm mask, at HV = 1100 V]{%
    \includegraphics[clip=true,trim=100 75 140 100,width=.245\textwidth,height=.15\textwidth]
                    {figures/GA0516_4b.pdf}}
  \subfloat[No mask, at HV = 1000 V]{%
    \includegraphics[clip=true,trim=100 75 140 100,width=.245\textwidth,height=.15\textwidth]
                    {figures/GA0516_4c.pdf}}
  \subfloat[No mask, at HV = 1100 V]{%
    \includegraphics[clip=true,trim=100 75 140 100,width=.245\textwidth,height=.15\textwidth]
                    {figures/GA0516_4d.pdf}}
  \caption{Same as Fig.~\ref{fig:GA0516_3}, but at the wheel position 1. 
    }
\label{fig:GA0516_4}
\end{figure*}


\begin{figure*}[!ht]
	\centering
	\includegraphics[width=0.9\textwidth,height=.55\textwidth]{figures/pavel_temp/LA2527_passport_temp.png}
	\caption{Illustration of the "MAPMT Passport" plots for one of the tubes, LA2527 (H12700). The standard six measurements included runs at three illumination settings (wheel positions 3, 4, and 6), each at two operating high voltage values (1000 V, and 1100 V).}
	\label{fig:LA2527_passport}
\end{figure*}
\begin{figure*}[!ht]
	\centering
	\includegraphics[width=0.9\textwidth,height=.55\textwidth]{figures/pavel_temp/LA2527_spectra_temp.png}
	\caption{Illustration of the "MAPMT Passport" plots for one of the tubes, LA2527 (H12700), continued. The standard six measurements included runs at three illumination settings (wheel positions 3, 4, and 6), each at two operating high voltages (1000 V, and 1100 V). Shown are the calculated SPE probability distribution functions $p_1(a)$, defined by the fit parameters resulting from the independent fitting procedures for each six settings. Blue color corresponds to the three sets at HV = 1000 V, and red - to the runs at HV = 1100 V. The parameters of the independent fits at three different illuminations result in very stable SPE shapes, essentially overlapping each other in the plots. The measurement functions $p_0(a)$ are shown as peaks around the pedestal at $a=0$ with the left sharp edge width corresponding to $\sigma$, and the right edge determined by the crosstalk.}
	\label{fig:LA2527_passport_spectra}
\end{figure*}

As a demonstration of the characterization procedure for the MAPMTs, Fig.~\ref{fig:CA7811}-\ref{fig:GA0516_4} show the measured signal amplitude probability distributions for one H8500 MAPMT pixel (CA7811, pixel 9) and one H12700 MAPMT pixel (GA0516, pixel 4) under various conditions, as well as their respective fit results. 
Fig.~\ref{fig:CA7811} and Fig.~\ref{fig:GA0516_1} illustrate the effect that the electronic cross talk from neighboring pixels has on the measured SPE fit parameters. 
We collected two sets of data intended to reduce the contribution of crosstalk from neighboring pixels. 
In the first (as described in Chapter 5) we used a black sheet of paper to mask all pixels on a single MAPMT, and punctured a 3mm hole over the pixel of interest (Fig.~\ref{fig:CA7811}a). 
However, with this setup one cannot fully characterize the unmasked pixel, as there is some dependence of the measured signal on the location of the incident photon. 
To provide full coverage of a single pixel\textquotesingle s surface, another set of measurements was taken with a 6mm x 6mm square hole cut out over a single pixel. 
With this configuration, the full face of the pixel of interest is illuminated, while the neighboring pixels remain mostly covered by the black paper. 
However, there is still non-negligible contribution from crosstalk with this configuration, due to imperfect alignment of the masks. 
This can be clearly seen in Fig.~\ref{fig:CA7811}b which shows the signal amplitude distribution with this 6mm x 6mm square hole cut out over pixel 9. 
One can see the contribution of the crosstalk appearing as a shoulder to the pedestal, albeit smaller than the crosstalk shoulder seen in Fig.~\ref{fig:CA7811}d where the full face of the MAPMT is illuminated. 

The resulting SPE fit parameters for Fig.~\ref{fig:CA7811}a-d indicate the inability of the model to fully describe the crosstalk in the H8500 MAPMTs. 
Most notably, in the data sets where the full-face of the MAPMT was illuminated (Fig.~\ref{fig:CA7811}c and Fig.~\ref{fig:CA7811}d) the $scale$ parameter changes by almost $7\%$ when the crosstalk is removed by the offline correlation analysis procedure compared to when it is kept in the data. 
Because the $scale$ parameter gives the average charge measured per photoelectron, it should be independent of the crosstalk. 
In contrast, we observe that the crosstalk in the H12700 MAPMTs can indeed be well described by the updated model, as is evident by comparing the fit parameters for Fig.~\ref{fig:GA0516_1}c and Fig.~\ref{fig:GA0516_1}d. 
All parameters are consistent between the two fits, despite the fact that the crosstalk was removed by the offline analysis prior to performing the fit for Fig.~\ref{fig:GA0516_1}c. 
This result exemplifies the ability of the model to extract the SPE parameters from the measured signal amplitude distributions in a crosstalk-independent manner. 

The sample comparison between typical H8500 and H12700 MAPMTs as shown in Figs.~\ref{fig:CA7811} and \ref{fig:GA0516_1} generally confirms our decision to switch to H12700 as the MAPMT of choice for the RICH detector. In the previous study (Ref.~\cite{DEGTIARENKO20171}), using different electronics front-end and data acquisition system, we observed that the values of the $\nu_{average}$ parameters are generally much smaller in H8500 than in H12700, leading to a significant difference in the expected efficiency of the tubes to SPE events, better for the H12700 MAPMTs. In the previous study the amplitude resolution was not good enough to uncover the additional difference between the two models: the crosstalk spectra are significantly wider in H8500, decreasing the expected SPE efficiency further, as compared to H12700. Wide crosstalk distributions in H8500 overlap noticeably with the shapes of the model SPE functions and do not allow the model to isolate them, while for H12700 MAPMTs the separation between the crosstalk and SPE distributions is reliable.  

The same sets of data were taken with the H12700 MAPMT high voltage set to 1100V to compare with the results of Fig.~\ref{fig:GA0516_1} which were taken at 1000V. 
The resulting amplitude probability distributions and fits are shown in Fig.~\ref{fig:GA0516_2}. 
As expected, both the $scale$ and $\nu_{average}$ parameters are larger when the HV is increased to 1100V, while the parameters describing the crosstalk, $\beta/\mu$ and $\zeta/scale$, are fairly consistent. 
Furthermore, by comparing Fig.~\ref{fig:GA0516_2}c and Fig.~\ref{fig:GA0516_2}d, we observe the same desirable characteristic that the SPE fit parameters are consistent with or without the offline removal of the crosstalk events from the data even at a larger high voltage setting.


Finally, Fig.~\ref{fig:GA0516_3} and Fig.~\ref{fig:GA0516_4} show the signal amplitude probability distributions for the same pixel on MAPMT GA0516 at higher illumination intensities. 
Specifically, Fig.~\ref{fig:GA0516_3} shows the results when the filter wheel is at position 2, for HV settings 1000V and 1100V, both with the full MAPMT face illuminated, and with the 6mm x 6mm square hole mask cutout applied. 
Comparing Fig.~\ref{fig:GA0516_3}c to Fig.~\ref{fig:GA0516_1}d (full-face illumination, 1000V), the $\mu$ parameter is almost a factor of 10 larger for the data collected with wheel position 2, but the characteristic parameters for the SPE response are consistent. 
The same can be said by comparing to the signal amplitude probability distribution in Fig.~\ref{fig:GA0516_4}c, which was measured at filter wheel position 1 (highest illumination). 
Even at roughly 100 times the light intensity, the resulting $scale$ parameter is consistent to what is measured at low light intensity. 

Fig.~\ref{fig:LA2527_passport} shows an example of the "Passport" plots obtained for a single MAPMT - in this case, an H12700 PMT labeled LA2527. 
Each plot shows different parameters extracted from the fits to the signal amplitude probability distributions vs. the pixel number, resulting in 64 data points per curve.
In all plots (excluding the top-right plot), the fit results are compared for the data taken with wheel positions 1, 2, and 3, and high voltages 1000V and 1100V (6 different configurations in total).
As expected the $scale$ and $\nu_{average}$ parameters are independent of the light intensity, but change with the applied high voltage. 
This is due to the increased amplification at each dynode at higher applied voltages.
The $\beta/\mu$ and $\zeta/scale$ parameters which describe the crosstalk from neighboring photoelectrons remain somewhat consistent between the different experimental configurations. 
However, the $\beta/\mu$ passport plot shows the dependence of the frequency of crosstalk on pixel location. For example, the first 8 and last 8 pixels all have significantly lower $\beta/\mu$ parameters. These pixels are along the edge of the MAPMT and therefore have (at least) one fewer neighboring pixel than those in the center of the MAPMT.
Consequently, the $\beta$ parameter for the amplitude probability distributions in these pixels is lower. 

The measurement of the absolute photon flux on each pixel was discussed in Chapter 5. The Quantum Efficiency (Q.E.) is obtained for each pixel by combining this measurement with the average number of photoelectrons measured per laser pulse, $\mu$, which is extracted separately for each pixel as a parameter of the fit to the signal amplitude probability distribution. The resulting Q.E. distribution is shown in the top-right plot of Fig.~\ref{fig:LA2527_passport}. These results indicate that on average the Q.E. for each pixel on the H12700 MAPMTs is about 21$\%$ for incident photons with wavelength 470nm.

The lower-right plot illustrates the quality of the SPE fit by showing the standard $\chi^2/NDF$ values for every fit, calculated for all bins in the measured spectrum with amplitudes above threshold. The accumulated number of events in each measured spectrum was very high and it is hard to expect an ideal model description with $\chi^2/NDF = 1$. The statistical quality of the fit was reasonably good for all measured spectra.

One final remark from the plots included in Fig.~\ref{fig:LA2527_passport} is that the SPE efficiency shown at the lower-left plot is slightly larger at 1100V than at 1000V. The efficiency was defined as the percent of SPE events above the threshold, which, in turn, was defined as the amplitude at which the number of events in the SPE distribution below the threshold is equal to the number of events in the crosstalk spectrum above it. The higher voltage leads to increased separation between the SPE spectra and the pedestal, corresponding to larger values of $\nu_{average}$ parameter, and thus increasing the efficiency. 

Fig.~\ref{fig:LA2527_passport_spectra} shows the extracted SPE functions for 9 pixels on the same MAPMT, again for all 6 configurations. The probability distributions are given as a function of the normalized charge amplitude, $a$. The functions extracted from the data measured with 1100V are noticeably more narrow around the peak than the data collected at 1000V, in agreement with the previously noticed differences between the values of $\nu_{average}$ parameters and efficiency at the different high voltages. The plots also illustrate the pedestal measurement functions around $a=0$ including the crosstalk contributions. The pedestal functions, and the SPE functions, measured independently at three illumination settings visibly overlap, and thus illustrate the stability of the fitting procedure and validate the applicability of the model, in its function to objectively extract the MAPMT characteristics.


\iffalse
The performance of MAPMTs was evaluated under certain high voltages.
The single photoelectron spectrum and the pixel efficiency for H12700 MAPMTs were tested and analyzed at 1000v, 1050v, 1075v, and 1100v.
The measurements performed at the reference supply voltage of 1000 V were compared to the measurements at different HV values in order to study the behavior of the MAPMT response as a function of the supply voltage.
As expected, it was found that the H12700 MAPMTs perform the best in the single photoelectron spectrum efficiency at higher voltages, especially at 1100v.
We see a significantly improved separation of the first photoelectron peak from the pedestal at higher voltage supplies (see~Fig.~\ref{fig:SPEhv}).
When the average deficiencies of the tested MAPMTs were analyzed, it was found that the average efficiency for 1000v, 1050v, 1075v and 1100v were approximately 4.6\%, 4.9\%, 5.0\%, and 5.2\%, respectively.
Therefore, the increase in detection efficiency is found to be over 10\% at 1100 V in comparison to 1000 V supply.
This separation is the crucial point for a single photon counting detectors such as CLAS12 RICH, where the occupancy is at the level of one photon per pixel.



It was also found that as the gain of the MAPMT increased, the efficiency ratio of 1100v to 1000v decreased.
The ratio is shown on~Fig.~\ref{fig:effratio} as a function of MAPMT gain reported by Hamamatsu.
The high voltage supply improves the performance of MAPMT dynode system, decreasing the fraction of the single photoelectron events below the pedestal peak.
This indicates that lower gain MAPMTs have a greater difference in the efficiencies at 1100v and 1000v, while higher gain MAPMTs have a smaller difference between the two voltage efficiencies.
The improvement for lower gain MAPMTs is more significant than for higher gain MAPMTs, because the high gain MAPMT has good separation of signal from pedestal even at the reference 1000 V.
Therefore, the low gain MAPMT benefit greatly from higher supply voltage.
The collected data are used to determine what high voltage the MAPMTs should be ran at to acquire the best results when the RICH detector is completed.


The parameters from Pavel's PMT response function are shown on the Fig.~\ref{fig:mu} and \ref{fig:characteristicPars} and correspond to the emission of the photoelectron ($\mu$), its collection and multiplication on the first dynode ($scale$ and $\nu$).
We have omitted other parameters that take into account the resolution effects of readout system or correspond to the cascade multiplication of the secondary electrons as they are out of scope focus of our analysis.
Given our requirements for single photoelectron signal sensitivity MAPMTs of our choice should be able to detect with high probability single photon that reaches photocathode.
In order to achieve this goal MAPMT should have high quantum efficiency of the photocathode as well as high collection efficiency of produced photoelectron combined with substantial signal multiplication to achieve good resolution of SPE peak.
These characteristics were studied in bulk for all 27520 channels (430x64) during the different light conditions and for different supplied HV values.


Fig.~\ref{fig:mu} shows the distribution of parameter $\mu$ which correspond to the average number of the photoelectrons produced during the measurements.
This parameters represents the convolution of the MAPMT photocathode quantum efficiency and laser setup light intensity.
Both characteristics should not depend on HV supply values and it is demonstrated on this figure by comparison of $\mu$ values extracted for the measurements at 1000 V and 1100 V.
However one can change the parameter by varying the laser light intensity which is evident from the left shift of $\mu$ values for lower light intensity measurements.

The other two free parameters are plotted on Fig.~\ref{fig:characteristicPars}.
They correspond to the average number of second-stage electrons produced on the first dynode by photoelectron (see~Fig.~\ref{fig:nu} and average signal amplitude for single photoelectron spectrum (see Fig.~{fig:scale}).
Both parameters characterize mainly the amplifying subsystem of MAPMT and therefore should depend on supplied HV.
The measurements at 1000 V and 1100 V confirm that amplification is improved at higher values of applied high voltage.
Traditionally the second parameter (gain) is often used to describe the amplification abilities of photomultipliers and often used in calibration and reconstruction procedures.
And it was shown that the extracted gains do not depend on the light conditions with a high degree of accuracy.
\fi
	  % Andrey
\section{Results}

This section reports on the study of 399 H12700 MAPMTs, acquired for the CLAS12 RICH2 detector upgrade. Each of them was tested in the same conditions by groups of six mounted in the Maroc tiles and irradiated simultaneously. The test procedure included six different setup conditions: two sets of applied high voltage (1000~V and 1100~V), and three laser light intensity settings at wheel positions 3, 4, and 6. The data were accumulated and pre-processed to make the non-linearity corrections and to convert the amplitudes into units of electric charge. After that the data were transferred to the ``parameterization factory'' computer workstation in which every accumulated spectrum was automatically analyzed and approximated with the 12-parameter fitting function, as was explained earlier. Each PMT was issued an ``MAPMT passport'' document listing the fit parameters for every measurement for all 64 anodes, showing the extracted SPE functions, and the parameter dependencies on pixel number, as illustrated in Figs.~\ref{fig:LA2527_passport} and \ref{fig:LA2527_passport_spectra}. The most important parameter extracted from the analysis for every pixel were i) $scale$, which measured the average charge collected at the anode from the single photoelectron events, ii) the average multiplicity $\mu$ of the photoelectrons per laser pulse, which can be converted to the quantum efficiency of the pixel when normalized to the calibrated incoming light in the pulse, iii) the calculated optimal threshold value for the separation of the single photoelectron events from the pedestal (including the crosstalk background), and iv) the corresponding estimate of the photodetection efficiency based on that value. The parameters of interest are also the characteristics of the photomultiplier, such as i) the gain on the first dynode evaluated in the model, ii) the amplitude width, and iii) the intensity of the crosstalk signal. The pedestal $\sigma$ parameter characterizes the quality of the Maroc measurement channel.

The six independent measurements in different conditions were used to verify the self-consistency of the results, using the model approximation features allowing the $scale$ parameter to be measured at various light conditions, ideally providing the same value, and similarly allowing the $\mu$ parameter (and hence the quantum efficiency) to be measured at various high voltages, also providing the same value. These features may be found in each of the ``MAPMT passports'', and they are also further illustrated in the following figures. \begin{figure}[h!]
	\centering
	\includegraphics[width=0.98\linewidth,trim=0 12 50 35,clip]{figures/pglobal_sc.pdf}
	\caption{Distribution of $scale$ (average charge per photoelectron) as determined by the fitting procedure for a set of 399 PMTs. All measured pixels contributed to the plots. Distributions measured at HV = 1000~V are shown by the solid lines, and those at HV = 1100~V by the dashed lines. The three colors correspond to the three different illuminations (essentially on top of each other).
	}
	\label{fig:pglobal_sc}
\end{figure}
Figure~\ref{fig:pglobal_sc} shows the distribution of the $scale$ parameter for the whole data set, separately for different high voltages and illumination settings. The distributions are clearly identical if obtained in different illuminations, and the change in high voltage is seen as an approximate multiplication of the $scale$ parameter by a factor about 2 when switching from 1000~V to 1100~V. Logarithmic $x$ scale in the plot helps to see the multiplication as a shift on the plot, roughly preserving the shape of the distribution. 

\begin{figure}[h!]
	\centering
	\includegraphics[width=0.98\linewidth, trim=0 12 50 35, clip]{figures/pglobal_Rs.pdf}
	\caption{Parameter $scale$ normalized to its average value over the three different illumination settings (wheel positions 3, 4, and 6).}
	\label{fig:pglobal_Rs}
\end{figure}
The stability and consistency of the fitting procedure is illustrated in Fig. \ref{fig:pglobal_Rs} in which every measured $scale$ parameter is normalized to the value of $scale$ averaged over the three measurements on the same pixel at the three different illuminations. The value of the ratio $R_{\mathrm{s}}$ serves as an estimate of the statistical uncertainty of the $scale$ evaluation procedure, and is approximately within 0.75\% for the tests at 1000~V, and within 0.5\% at 1100~V 

In the bulk measurements, one PMT was measured in one Maroc location. To be confident that different Maroc locations do not systematically contribute to the differences between the PMTs, we compared all six locations by making the standard sets of measurements using six PMTs in six runs in which every PMT occupied each of the six Maroc positions in turn, and compared the extracted parameters for every pixel made six times in the different locations. One of the results of such a comparison is shown in Fig.~\ref{fig:R_scale_maroc_avg}. 
\begin{figure}[h!]
	\centering
	\includegraphics[width=0.98\linewidth, trim=0 10 15 10, clip]{figures/R_scale_maroc_avg.png}
	\caption{Evaluated precision of the scale parameter measurement for the two high voltage settings.}
	\label{fig:R_scale_maroc_avg}
\end{figure}
The histograms show the distributions of the ratios of the measured $scale$ parameter to the average of its values measured in the six Maroc locations. The spreads observed are different for the runs at 1000~V and at 1100~V, and the values are comparable to the spreads observed in Fig.~\ref{fig:pglobal_Rs}. Thus we conclude that switching the location of the PMT in the test setup did not cause significant systematic uncertainties in the measured parameters. Similar studies were performed for the other extracted parameters. The observed stability of the extracted quantum efficiencies during these tests, and also comparisons of measurements of quantum efficiency on the same MAPMT made few months apart, indicated to the short- and long-term stability of the laser light source yield at a very good level within the range of statistical errors in the evaluated $\mu$ parameter.  


\begin{figure}[h!]
	\centering
	\includegraphics[width=0.98\linewidth,trim=0 12 50 35,clip]{figures/pglobal_qe_all.pdf}
	\caption{Distribution of $\mu$ divided by the measured number of photons per pulse at wheel position 3. All measured pixels contributed to the plots. Distributions measured at HV = 1000 V are shown in blue, the ones at HV = 1100 V in red (essentially on top of each other). The three line styles (dotted, dashed, and solid) correspond to different illuminations. For the data collected at wheel position 3, this ratio is the quantum efficiency of the individual pixels.}
	\label{fig:pglobal_qe_all}
\end{figure}
Figure~\ref{fig:pglobal_qe_all} shows a pattern similar to Fig.~\ref{fig:pglobal_sc} for the $\mu$ parameter, with the difference that $\mu$ essentially does not depend on high voltage, but it is proportional to the light intensity. The plot shows that the distributions at different high voltages are on top of each other at a given light intensity but shift in log scale when the light intensity changes. In the plot, the parameter $\mu$ is shown normalized to the number of photons coming to each pixel in the ``wheel position 3'' setting, to provide the associated value of quantum efficiency. The overall averaged quantum efficiency measured in this work at the wavelength of 470 nm is close to the values given in the manufacturer's specifications for the H12700 MAPMTs \cite{H12700}.

\begin{figure}[h!]
	\centering
	\includegraphics[width=0.98\linewidth, trim=0 12 50 35, clip]{figures/pglobal_mHV.pdf}
	\caption{The ratio of the $\mu$ parameters from the fit results at HV = 1100 V to the results at HV = 1000 V.}
	\label{fig:pglobal_mHV}
\end{figure}
Figure~\ref{fig:pglobal_mHV} illustrates the stability of the evaluated $\mu$ parameter measured at different values of high voltage. As we had only two settings, the plot shows the distributions of the ratios $R_{\mu\mathrm{HV}} = \mu_{\mathrm{HV1.1}}/\mu_{\mathrm{HV1.0}}$ of the values of $\mu$ measured at 1100~V to the values at 1000~V. The width of the distribution around $R = 1$ may characterize the statistical uncertainty in the measurement of $\mu$. The plot shows that the relative $\mu$ spread is approximately within 1\% of the value. In first approximation, the quantum efficiency is not expected to be dependent on the high voltage applied to a PMT. However, the distributions show slight systematic shifts in the ratio, indicating a small dependence of quantum efficiency on the high voltage applied, with a slope of about 0.2\% per 100 V change. Practically the change is insignificant and within the statistical uncertainties, however, there might be some attempts to explain it assuming, for example, that the larger electric field at the cathode region may improve the probability of photoelectron knock out, or improve the collection probability of the photoelectrons at the first dynodes.

\begin{figure}[h!]
	\centering
	\includegraphics[width=0.98\linewidth, trim=0 12 50 35, clip]{figures/pglobal_eff.pdf}
	\caption{Distribution of the measured efficiency for all pixels at wheel position 4.}
	\label{fig:pglobal_eff}
\end{figure}
Figure~\ref{fig:pglobal_eff} shows the estimated values of the photodetection efficiency based on the calculated optimal threshold value for the separation of the single photoelectron events from pedestal (including the crosstalk background). The calculation for every pixel was performed for the measurements at the lowest illumination settings at wheel position 4, when both parameters $\mu$ and $\beta$ are small and the probability of having two crosstalk electrons in one event was negligible. Such a condition imitates the real operations of the MAPMTs in the RICH detector in the best way, as the number of photons from one relativistic particle is expected to be small. The figure also illustrates the generally very high (above 96\%) single photon efficiency of all tested H12700 PMTs at the planned operational high voltage value of 1000~V. The efficiency is improved significantly at 1100~V, with the value of ``inefficiency'' decreasing by approximately a factor of 2 in these conditions.

\begin{figure}[h!]
	\centering
	\includegraphics[width=0.98\linewidth, trim=0 12 50 35,clip]{figures/pglobal_nu.pdf}
	\caption{Distribution of $\nu$ (average gain on first dynode) as determined by the fitting procedure for a set of 399 PMTs.}
	\label{fig:pglobal_nu}
\end{figure}
The efficiency improvements at larger high voltage are correlated with the observed increases of the average degree of multiplication of the photoelectrons on the first dynodes of the MAPMTs. The average gain $\nu$ is evaluated in the model using the five parameters describing the shapes of the SPE amplitude distributions. The average gain is clearly dependent on the energy acquired by the photoelectron traveling from the photocathode to the first dynode. The spread in this parameter over the whole data set is noticeable, but the systematic increase at 1100~V is quite prominent, as shown in Fig.~\ref{fig:pglobal_nu}. This figure further illustrates the consistency and stability of the fitting procedure as the distributions built for different illuminating conditions are very close to each other.

\begin{figure}[h!]
	\centering
	\includegraphics[width=0.98\linewidth, trim=0 12 50 35, clip]{figures/pglobal_Rn.pdf}
	\caption{Parameter $\nu$ normalized to its average value over the three different illumination settings (wheel positions 3, 4, and 6).}
	\label{fig:pglobal_Rn}
\end{figure}
Figure~\ref{fig:pglobal_Rn} is similar to Fig.~\ref{fig:pglobal_Rs}, showing the measured $\nu$ parameters normalized to the value of $\nu$ averaged over the three measurements on the same pixel at the three different illuminations. The value of the ratio $R_{\nu}$ serves as an estimate of the statistical uncertainty of the $\nu$ evaluation procedure, and is approximately within 5\%. The distribution is visibly non-Gaussian as $\nu$ is a complicated function of five variable signal shape parameters in the fit. There is a small difference between the distributions at different high voltage settings.

\begin{figure*}[t!]
	\centering
	\begin{subfigure}[c]{0.48\linewidth}
		\centering
		\includegraphics[width=\linewidth]{figures/pglobal_sc2d.pdf}
		\caption{Scale, HV = 1.0 kV (pC per 1 photoelectron)}
		\vspace{0mm}
	\end{subfigure}%%
	\begin{subfigure}[c]{0.48\linewidth}
		\centering
		\includegraphics[width=\linewidth]{figures/pglobal_qe.pdf}
		\caption{Quantum Efficiency (percent)}
		\vspace{0mm}
	\end{subfigure}%%
	\vspace{3mm}
	\begin{subfigure}[c]{0.48\linewidth}
		\centering
		\includegraphics[width=\linewidth]{figures/pglobal_beta.png}
		\caption{Crosstalk relative to $\mu$}
		\vspace{0mm}
	\end{subfigure}%%
	\begin{subfigure}[c]{0.48\linewidth}
		\centering
		\includegraphics[width=\linewidth]{figures/pglobal_eff2d.pdf}
		\caption{Efficiency, wheel position 3, HV = 1.0 kV (percent)}
		\vspace{0mm}
	\end{subfigure}%%
	\caption{Two dimensional plots showing the average (a) scale, (b) quantum efficiency, (c) crosstalk relative to $\mu$, and (d) efficiency as a function of pixel location. The results are averaged for the full set of 399 Hamamatsu H12700 MAPMTs. The pixel numbers increment from left to right, top to bottom, with pixel \#1 in the top left corner.}
	\label{fig:2d_avg_fit_results}
\end{figure*}


Figure~\ref{fig:2d_avg_fit_results} illustrates the dependencies of several major parameters on the pixel number for the full set of MAPMTs studied, including the average amplitude of the single photon amplitude $scale$, Quantum Efficiency, the relative probability of the crosstalk events $\beta/\mu$, and the evaluated efficiency. Generally, the set exhibits a very good uniformity of the average parameters, much smaller than the spreads observed between pixels in a single MAPMT or between the tubes. The Quantum Efficiency is slightly higher at the edges of the MAPMT and still higher at the corners (larger areas of the border pixels are taken into account in the QE calculation). The crosstalk probability pattern is consistent with the hypothesis that it is dependent on the number of neighbors: it is smaller at the edges, and still smaller in the corners of the MAPMT. The four outliers in pixels 16, 24, 32, and 40 are most likely due to the feature of all Maroc boards used, exhibiting significantly wider pedestals in these pixels, hiding the crosstalk under the pedestal Gaussian and causing the fitting procedure to fail to fit the crosstalk properly. The average efficiency pattern shows somewhat better values in columns 4 and 8, likely correlated with the widths of the crosstalk contributions.    



The parameter database accumulated as the result of this work was used for the selection of the MAPMTs for installation in the RICH detector, and for the optimization of the future run parameters, such as the tube placement selection, as well as setting the values of operating high voltage, electronics gains, and thresholds in the detector.


The data also provide the opportunity to evaluate the spread of such parameters in the mass production of the MAPMT devices as the channel gains, quantum efficiencies, SPE spectral shapes, and parameters of the crosstalk, - across the face of each tube, and across the whole set. The results show that the quality of MAPMT mass production at Hamamatsu is high and satisfies our needs in good quality single photoelectron detection.



%We saw that although there is little difference in crosstalk signals, the H12700 PMTs suffer less from dark current, have narrower SPE spectra, and have higher $\mu$ and relative efficiency values.
%An example plot of the $\mu$'s and relative efficiencies of H8500 and H12700 PMTs with similar low and high gains is shown in Figure \ref{efficiency}.

%We see that the relative efficiency is closely related to the $\mu$ which is on average, over all pixels at all voltages for all the PMTs we tested, $29\pm5$ percent higher in H12700 than H8500 MAPMTs. One concern with these $\mu$ measurements however is that the laser system used to measure these PMTs was only incident on a portion of each pixel, consequently missing their sum total effect and pinpointing possible spatial dependencies which should be further studied and perhaps remeasured with a fully illuminated MAPMT instead of collimated pinpoint laser light. In terms of crosstalk for the two varieties of MAPMT the H12700s appear to be better than the H8500s. The H12700s have a decrease in crosstalk by nearly a factor of two. Additional studies of dark current in the H12700s would be useful, as the dark current is usually dominated by individual pixels or bad regions of the PMT instead of spread around evenly like in the H8500s, but overall the two varieties are not very different in terms of dark current.
			  % All
\section{Conclusion}
We have tested several hundred Hamamatsu H12700 and H8500 multi-anode photomultiplier tubes in order to evaluate the application of these MAPMTs in the Ring Imaging Cherenkov detectors. The PMT’s performance study was done using the laser stand with low light illumination to simulate one-photoelectron regime as expected in the real experiment. We were using the state-of-the-art mathematical model for the characterization of the photomultipliers that were design to describe the PMT’s signals with low number of photoelectrons. 
The model was modified to take into account the effects of the cross-talk between the neighboring pixels. 
The stability of the fit parameters was proven using measurements in different laser beam intensities and at different applied high voltages. Thus, the model allowed us to characterize the photomultipliers independently of the test measurement conditions.
We carefully studied the critical for the experiments PMT's properties for the H12700 and H8500 Hamamatsu models:
the shapes of the one photoelectron spectra,
PMT's gain as a function of high voltage,
efficiency to detect one-photoelectron signal,
amplification of the first dynode that is critical for the PMT efficiency working in the one-photoelectron regime,
cross-talk between different pixels and
quantum efficiency. All these parameters were extracted for each PMT pixel.

The large amount of tested  PMT’s gives us the possibility to do statistical analysis of all parameters in hands and compare the performance of H12700 and H8500 for an application in Cherenkov detectors. The parameter database accumulated as the result of this work will be used for the selection of the MAPMTs for installation in the RICH detector, and for the optimization of the future run parameters, such as the tube placement selection, setting the values of operating high voltage, gains and thresholds of the front-end electronics. 

This data also makes it possible to estimate the spread of such parameters in the mass production of MAPMT devices as channel gain, quantum efficiency, SPE spectral shapes, crosstalk parameters between the pixels of one PMT and throughout the detector.
The results show that the quality of MAPMT mass production at Hamamatsu is high and satisfies our needs in the good quality single photoelectron detection.
 


		  % All
%%%%%%%%%%%%%%%%%%%%%%%%%%%%%%%%%%%%%%%%%%%%%%%%%%%%%%%%%%%%%%%%%%%%%%%%%%

\bibliographystyle{elsarticle-num-names}
\bibliography{RICH}

\newpage
\newpage
\appendix
\section{}
\label{AppendixA}

In the case of H12700 MAPMTs the signal amplitudes of the crosstalk contributions from different neighboring pixels were found to be relatively small and similar to each other, allowing us to use in the model the single averaged spectral term for all neighbors of a given pixel. Every crosstalk contribution comes from a single electron in one of the neighboring pixels, their average number in one measurement $\beta$ is expected to be comparable with $\mu$, and multiple crosstalk events in one measurement happen independently. That means that the probability of observing $i$ crosstalk contributions in one event is distributed according to the Poisson distribution
\begin{equation}
\label{CTPoisson}
 P(i;\beta) = \frac{\beta^{i} e^{-\beta}}{i!}.
\end{equation}
Poisson-like shapes of the general SPE distribution functions suggest the shape of the crosstalk contribution in the form of a Poisson distribution, scaled to represent the portion of the charge generated in the neighboring pixel, transferred to the pixel studied. The representation of such distribution for one crosstalk electron takes the form
\begin{equation}
\label{C1Poisson}
 C_1(j) = P(j;\lambda) = \frac{\lambda^{j} e^{-\lambda}}{j!},
\end{equation}
where $j$ is the non-negative integer, corresponding to the amplitude values $a_j = j \zeta / \lambda $, relating the discrete Poisson scale to the set of $a$ values such that the average crosstalk contribution to the measurement function from one crosstalk electron was equal to the value of $\zeta$ parameter (the average $\langle j \rangle$ in Eq.~(\ref{C1Poisson}) equals to $\lambda$).

Corresponding distributions for the events with $i$ crosstalk electrons then take the form of convolution powers, which can be explicitly calculated in the case of Poisson distributions: 
\begin{equation}
\label{CiPoisson}
 C_i(j) = C_1^{*i}(j) = P(j;i\lambda).
\end{equation}

Thus, similar to Eq. (13) in Ref.~\cite{DEGTIARENKO20171}, the discrete distribution can be represented as a function of the normalized amplitude $a$ in the form of the infinite sum of correspondingly weighted delta-functions, one per each value of $j \geq 0$:
\begin{equation}
\label{deltaf}
  D_{ct}(a)= \sum\limits_{j=0}^{\infty}
  \delta \left (a - \frac{j\zeta}{\lambda} \right )
   \sum\limits_{i=0}^{\infty} P(i;\beta) C_i(j) .
\end{equation}
The convolution of this distribution with the Gaussian measurement function (sigma equal to $\sigma_a$) will result in the continuous function similar to Eq.~(15) in~\cite{DEGTIARENKO20171}:
\begin{equation}
\label{RCmodel}
  R_{ct}(a)= \sum\limits_{j=0}^{\infty} \frac{1}{\sqrt{2 \pi} \ \sigma_a} 
  \exp{\left [- \frac{(a - j \zeta / \lambda)^{2}}{2 
\ \sigma_a^{2}} \right ]}
   \sum\limits_{i=0}^{\infty} P(i;\beta) C_i(j) .
\end{equation}

The new function $R_{ct}(a)$, parametrically dependent on $\sigma_a$, $\beta$, $\zeta$, $\lambda$, describes the effective measurement function applied to every signal. Recorded signals are the results of the convolution with this function. In particular, in the events with no photoelectrons ($m=0$), the pedestal distribution takes the form of $R_{ct}(a)$. For a given set of parameters the function $R_{ct}(a)$ is evaluated numerically in the model implementation and then used in the calculations as described in Ref.~\cite{DEGTIARENKO20171}, by replacing the measurement function $R(a)$ with $R_{ct}(a)$ in convolution with $D(a)$ function in Eq.~(14) in Ref.~\cite{DEGTIARENKO20171}. The function $D(a)$ as defined in Eq.~(13), Ref.~\cite{DEGTIARENKO20171}, much like the function $D_{ct}(a)$ in Eq.~(\ref{deltaf}) in this work, represents the infinite set of delta-functions, and the convolution calculation just needs the values of the tabulated function $R_{ct}(a)$ in all the final sums. The new equivalent for Eq.~(16) in Ref.~\cite{DEGTIARENKO20171} is thus
\begin{equation}
\label{GCmodel}
  G_{ct}(a,n;\sigma_a,\beta,\zeta,\lambda)= R_{ct}(a-n/\nu;\sigma_a,\beta,\zeta,\lambda).
\end{equation}
The new function $G_{ct}(a,n;\sigma_{\mathrm{eff}},\beta,\zeta,\lambda)$ is then used to replace the function $G(a,n;\sigma_{\mathrm{eff}})$ in the final model equation, Eq.~(36) in Ref.~\cite{DEGTIARENKO20171}, keeping the same form. The change is that instead of being a standard Gaussian, the measurement function is now distorted by the crosstalk contribution, requiring three extra parameters to approximate the data.
		  % All

\end{document}
